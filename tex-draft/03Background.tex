
\section{Background}

\subsection{Quantum, Linear Algebra and Dirac Notation in a Nutshell}
\begin{itemize}
  \item Axioms of Linear Algebra

  \item Tensor Product Extension

  \item Dirac Notation interpretation, ambiguity and associativity
\end{itemize}

\subsection{Term Rewriting}

\begin{itemize}
  \item Universal Algebra

  \item TRS and equational theory
  
  \item Important properties: confluence and termination
  
  \item Variant: AC rewriting
  
  \item Modularity
\end{itemize}

When the whole formal system becomes large ane complicated, we can slice it into different layers. From such a point of view, different aspects of the system, i.e. language syntax, semantical interpretation, rewriting rules and proofs, are built in the bottom-up style. For example, in our work there will be two foundational systems for complex numbers and atomic basis, the core and extended Dirac notations are then constructed subsequently. Such a layered system provides flexibility and generality w.r.t. the basic modules. Also, properties can be developed by composing the corresponding proofs from the submodules. In our work, the local confluence of the core Dirac notation is proved by such modularity utilizing the avatar lemma, which is discussed in Sec.\ref{sec: conf-modular}.