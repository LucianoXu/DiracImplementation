
\section{Motivating Example}

When the state of a many-body quantum system cannot be represented by the direct product its component states, we say it is in an entangled state. Entangled quantum states are important resources for quantum computing and quantum information, as they demonstrate the nonlocal correlation among the subsystems. 
Here is such a well-known property among others: for a maximally entangled two-body system $\ket{\Phi}$, if we apply operation $M$ on one subsystem, the result state is the same as applying $M^\top$ on the other entangled subsystem. 
A subsequent corollary is, if we apply operations $M_1$ and $M_2$ respectively on two subsystems of a maximally entangled state, the overlap of the result state on the origin will be the same as the trace of operation $M_1^\top M_2$ on only one subsystem. In the following, we take the corollary as an example, rephrase it in the formal language called \textit{Dirac notation},
% It serves as a good motivating example showing the key techniques and difficulties, and 
and show how our theory and tool helps deciding the statement.

\begin{example}
  Let $\ket{\Phi} = \sum_{i \in V}\ket{i, i}$ be the maximally entangled state on Hilbert space $\mathcal{H}_V\otimes\mathcal{H}_V$. Then for all operators $M_1, M_2 \in \mathrm{Hom}(\mathcal{H}_V, \mathcal{H}_V)$,
  $$
  \bra{\Phi} (I \otimes M_2) (M_1 \otimes I) \ket{\Phi} = \mathrm{tr}(M_2^\top M_1).
  $$
\end{example}

Dirac notation is widely used to describe quantum states and operations. It uses the braket notation to express linear algebra concisely, while the latter is the foundational language for quantum.
In Dirac notation, (pure) states are denoted as \textit{kets}, representing vectors in the Hilbert spaces. The state $\ket{\Phi}$ is such an example. And because the entangled state consists of two subsystems $\mathcal{H}_V$, the overall state space is the tensor product $\mathcal{H}_V \otimes \mathcal{H}_V$ of subsystem spaces.
Therefore, we have $\ket{\Psi} = \sum_{i\in V} \ket{i} \otimes \ket{i}$. Here $i$ is summed over all orthogonal basis in $V$, corresponding to the \textit{maximal} entanglement. Because of the natural isomorphism of tensor product spaces, the state $\ket{\Phi}$ can also be written as $\sum_{i \in V} \ket{i, i}$. 
Further on, operations on states are linear operators in the corresponding spaces. Here $M_1 \otimes I$ means the operation $M_1$ is applied on the first subsystem, with an identical operation $I$ applied on the second one. The notation $I \otimes M_2$ is interpreted similarly. The successive multiplication $(I\otimes M_1)(M_2 \otimes I)\ket{\Phi}$ means the sequential application of two operations on the entangled state.
Finally, we take the inner product of two states $\ket{u}$ and $\ket{v}$. To express the inner product operation in an associative way, the \textit{dual space} $\mathcal{H}_V^*$ is introduced, whose elements are linear functionals, denoted as \textit{bras} like $\bra{v}$. The left hand side of the equation computes the inner product of the post-operation state with the original maximally entangled one. The right hand side is the trace of associated operation $M_2^\top M_1$.

Here is a step-wise proof of the equivalence by human.
\begin{align}
    & \ \bra{\Phi} (I \otimes M_2) (M_1 \otimes I) \ket{\Phi} \notag \\
    \text{\{ expand $\ket{\Phi}$, absorbing identity operators \}} & = \left(\sum_{i\in V} \bra{i} \otimes \bra{i}\right) M_1 \otimes M_2 \left(\sum_{i \in V} \ket{i} \otimes \ket{i}\right) \\
    \text{\{ push terms into the big operator \}} & = \sum_{i\in V} \sum_{j \in V} \left( \bra{i} \otimes \bra{i} \cdot  M_1 \otimes M_2 \cdot \ket{j} \otimes \ket{j} \right) \\
    \text{\{ rearrange operations on two subsystems \}} & = \sum_{i\in V} \sum_{j \in V} \bra{i} M_1 \ket{j} \times \bra{i} M_2 \ket{j}
\end{align}

\begin{align}
  & \ \mathrm{tr}(M_2^\top M_1) \notag \\
  \text{\{ expand trace definition \}} & = \sum_{k \in V} \bra{k} M_2^\top M_1 \ket{k} \\
  \text{\{ insert the identity operator \}} & = \sum_{k \in V} \bra{k} M_2^\top (\sum_{l \in V} \ket{l}\bra{l}) M_1 \ket{k} \\
  \text{\{ push terms into the big operator \}} & = \sum_{k \in V} \sum_{l \in V} \bra{k} M_2^\top \ket{l}\bra{l} M_1 \ket{k} \\
  \text{\{ simplify the transpose operator \}} & = \sum_{k \in V} \sum_{l \in V} \bra{l} M_2 \ket{k} \times \bra{l} M_1 \ket{k}
\end{align}

The human proof proceeds by rewriting both sides of the equation according to the laws of linear algebra. 
However, manual proofs take much human effort, and can be error-prone in large-scale examples. 
Therefore, a natural goal is to design the formal verification, or even the decision procedure, based on Dirac notations.

The example illustrates the possibility and challenge for automated deduction of Dirac notations.
On the one hand, some of the rewritings can be concluded as axioms, and the rewritings appear to be simplifying the terms. For example, $(I \otimes M_2)(M_1 \otimes I) = M_1 \otimes M_2$ in Eq.(1), and $\bra{k}M_2^\top\ket{l} = \bra{l}M_2\ket{k}$ in Eq.(7). On the other hand, some techniques require careful scheduling (e.g., rearranging $\bra{i}\otimes\bra{i}\cdot M_1 \otimes M_2 \cdot \ket{j} \otimes \ket{j}$ in Eq.(3)), and some seems to be heuristics (e.g., inserting the identity operator $\sum_{l \in V}\ket{l}\bra{l}$ in Eq.(5)).

Our work resolves the decision problem of Dirac notations by designing a \textit{term rewriting system} (referred as TRS in the following). The general idea is to view Dirac notation as a universal algebra, summerize the equational axioms and specify the direction of rewriting. The proof by such a TRS can be explained and checked, and we can prove theoretical properties for the system, such as termination, confluence and even completeness.