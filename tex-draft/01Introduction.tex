
\section{Introduction}
Storyline:
\begin{enumerate}
    \item Formal method is an important topic in quantum computing with lots of applications.
    \item Automation is a central topic in logic and formal methods. It reduces human efforts, enables reasoning in large scales, and is critical to the performance and usability of tools.
    \item The current results in formalizing quantum computing are divided into two types: automated tools targeting at special cases (QBricks, QEC, EasyPQC), and expressive foundational formalizations (CoqQ). The automation level is still low for general formalization.
    \item Our work explores the general automation of quantum formal methods.
    \item Dirac notation is a good starting point for generality.
    \item For automation, term rewriting system is a promising method. It is explainable with theoretical guarantees.
\end{enumerate}


(By GPT4:)

Formal methods constitute a crucial aspect of quantum computing, offering a robust framework for verifying and validating quantum algorithms and systems. These methods have found applications in diverse areas, from cryptography to error correction, highlighting their essential role in advancing the field. Within formal methods, automation emerges as a pivotal topic. Automated tools significantly reduce human effort, enable reasoning at larger scales, and enhance the performance of verification tools, thereby facilitating more efficient and accurate formalization processes.

Current advancements in formalizing quantum computing can be broadly categorized into two groups: specialized automated tools and foundational formalizations. Examples of automated tools targeting specific scenarios include QBricks, QEC, and EasyPQC, which are designed for particular use cases but lack general applicability. On the other hand, expressive foundational formalizations such as CoqQ offer a more comprehensive framework but exhibit limited automation capabilities.

Despite these developments, the level of automation in general formalizations remains insufficient. Addressing this gap, our research focuses on advancing the automation of quantum formal methods, thereby extending their applicability and utility. We identify Dirac notation as an ideal starting point due to its generality and widespread use in quantum mechanics.

To achieve automation, we propose employing a term rewriting system, which shows promise in systematically handling complex transformations and ensuring correctness. Our work aims to develop a robust framework for the automated formalization of quantum computing, leveraging the strengths of Dirac notation and term rewriting systems to achieve higher levels of automation and efficiency in this critical field.


Related works.

Big Operator.\cite{Bertot2008}

Lineal.\cite{Arrighi2017}

CoqQ.\cite{Zhou2022}

Decidability of Linear algebra.\cite{Solovay2012}

CiME.\cite{Contejean2011}

Term Rewriting. \cite{Baader1998}

Termination techniques. \cite{Arts2000} \cite{Giesl2002} \cite{Giesl2006}

Quantum. \cite{Nielsen2010}

Wolfram. \cite{WolframLanguage}

\vspace{2em}

Our contribution:
\begin{itemize}
  \item \textbf{The first theory for the decision procedure of Dirac notations.} 
  We rephrased the formal language of Dirac notation and developed its decision procedure. The language is separated into the core language of basic symbols and the extension with big operator sum. We proved the soundness and relative completeness (?) of the decision procedure, which is formally verified in \texttt{Coq}. Moreover, the core language is a pure term rewriting system with local confluence proved.
  \item \textbf{An efficient and powerful tool implementation.}
  We provide the implementation of the language and decision procedure in Mathematica. We encoded a broad scope of practical examples, and all the encoded problems can be automatically checked efficiently.
\end{itemize}
