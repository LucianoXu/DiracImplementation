\chapter{20240105}


\section{Dirac Notation}


\subsection{Syntax and Typing}

\begin{postulate}[complex number]
  $\mathbb{C}$ is an algebra for complex numbers. It has the symbols $(0, 1, \textrm{add}, \textrm{mul}, \textrm{conj})$.
\end{postulate}
It means that somehow we can express and decide complex number terms, but the theory should not be considered here.

\begin{definition}[atomic type and term]
  The atomic types are $\mathbf{Z}_n$.
  The constants of the type $\mathbf{Z}_n$ is $\{i \in \mathbb{N} : i<n\}$
  Here $n$ and $i$ are natural numbers.
\end{definition}

% \textbf{Remark:} We define atomic types separatedly because they describe the types of atomic quantum subsystems.

\newcommand*{\unit}{\texttt{unit}}
\newcommand*{\utt}{\texttt{tt}}
\newcommand*{\fst}{\texttt{fst}}
\newcommand*{\snd}{\texttt{snd}}
\newcommand*{\reduce}{\ \triangleright\ }

\begin{definition}[syntax]
  \begin{align*}
    \tau ::= \ 
      & T\ |\ \mathbb{C}                && \text{(basic types)} \\
      & |\ (\tau * \tau)                && \text{(product type)} \\
      % We use product type instead of tuple type because we want to adhere to the strict criterion of equivalence for tensor product: associativity is not considered.
      & |\ \unit                          && \text{(unit type)} \\
      % product type will not be squashed (i.e., tau*unit != tau) also because the strict criterion of equivalence for tensor product.
      & |\ [\tau, \tau]                   && \text{(Dirac notation types)} \\ 
    e ::= \ 
      &    x                          && \text{(variable)} \\
      & |\ 0\ |\ 1\ |\ \mathrm{add}(e, e)\ |\ \mathrm{mul}(e, e)\ |\ \mathrm{conj}(e) && \text{(complex number)} \\
      & |\ c                          && \text{(constant)} \\
      & |\ (e, e)                     && \text{(pair)} \\
      & |\ \utt                       && \text{(inhabitant of \unit)} \\
      & |\ \fst\ e\ |\ \snd\ e        && \text{(projection)} \\
      & |\ \mathbf{0}_{\tau, \tau}    && \text{(zero operator)}\\
      & |\ \mathbf{1}                 && \text{(identity operator)}\\
      & |\ \delta_{e, e}.e            && \text{(Delta expression)} \\
      & |\ \ket{e}                    && \text{(ket)} \\
      & |\ \bra{e}                    && \text{(bra)} \\
      & |\ e^*                        && \text{(conjugate)} \\
      & |\ e^T                        && \text{(transpose)} \\
      & |\ \alpha.e                   && \text{(scaling)} \\
      & |\ e + e                      && \text{(addition)} \\
      & |\ e \cdot e                  && \text{(multiplication)} \\
      & |\ e \otimes e                && \text{(tensor product)}
    \end{align*}
  Here $T$ is a basic type, $x$ is a variable and $c$ is a constant of some basic type $T$. $\tau$ and $e$ are called types and terms respectively.
\end{definition}

\begin{definition}[typing rules]
  A typing assumption has the form $x : \tau$, meaning variable $x$ has the type $\tau$. A typing context $\Gamma$ consists of typing assumptions and each variable appears only once at most.

  A typing judgement $\Gamma \vdash e : \sigma$ indicates that $e$ is a term of type $\sigma$ in context $\Gamma$. The well-typed terms are defined by the following rules:
  \begin{gather*}
    \frac{x : \sigma \in \Gamma}{\Gamma \vdash x : \sigma}
    \qquad \frac{c\ \textrm{is a constant of}\ T}{\Gamma \vdash c : T}\\
    \\
    \frac{\Gamma \vdash e_1 : \tau \qquad \Gamma \vdash e_2 : \sigma}{\Gamma \vdash (e_1, e_2) : ( \tau * \sigma )} \\
    \\
    \frac{\Gamma \vdash e : ( \tau * \sigma ) \qquad \tau \neq \unit \qquad \sigma \neq \unit}{\Gamma \vdash \fst\ e : \tau}
    \qquad
    \frac{\Gamma \vdash e : ( \tau * \sigma ) \qquad \tau \neq \unit \qquad \sigma \neq \unit}{\Gamma \vdash \snd\ e : \sigma} \\
    \\
    \frac{}{\Gamma \vdash \utt : \unit}\\
    \ \\
    \frac{}{\Gamma \vdash \mathbf{0}_{\tau, \sigma} : [\tau, \sigma]}
    \qquad
    \frac{}{\Gamma \vdash \mathbf{1} : [\unit, \unit]}
    \qquad
    \frac{\Gamma \vdash s : \rho \qquad \Gamma \vdash t : \rho \qquad \Gamma \vdash e : [\tau, \sigma]}{\Gamma \vdash \delta_{s, t}.e : [\tau, \sigma]}\\
    \ \\
    \frac{\Gamma \vdash t : \tau}{\Gamma \vdash \ket{t} : [\unit, \tau]}
    \qquad 
    \frac{\Gamma \vdash t : \tau}{\Gamma \vdash \bra{t} : [\tau, \unit]}\\
    \ \\
    \frac{\Gamma \vdash e : [\tau, \sigma]}{\Gamma \vdash e^* : [\tau, \sigma]}
    \qquad
    \frac{\Gamma \vdash e : [\tau, \sigma]}{\Gamma \vdash e^T : [\sigma, \tau]} 
    \qquad
    \frac{\Gamma \vdash \alpha : \mathbb{C} \qquad \Gamma \vdash e : [\tau, \sigma]}{\Gamma \vdash \alpha.e : [\tau, \sigma]}\\
    \ \\
    \frac{\Gamma \vdash e_1 : [\tau, \sigma] \qquad \Gamma \vdash e_2 : [\tau, \sigma] }{\Gamma \vdash e_1 + e_2 : [\tau, \sigma] }
    \qquad 
    \frac{\Gamma \vdash e_1 : [\tau, \rho] \qquad \Gamma \vdash e_2 : [\rho, \sigma] }{\Gamma \vdash e_2 \cdot e_1 : [\tau, \sigma] } \\
    \ \\
    \frac{\Gamma \vdash e_1 : [\tau, \sigma] \qquad \Gamma \vdash e_2 : [\tau', \sigma'] }{\Gamma \vdash e_1 \otimes e_2 : [(\tau * \tau'), (\sigma * \sigma')]}
  \end{gather*}
\end{definition}

\textbf{Remark:} We say a term $e$ is a Dirac notation if $\Gamma \vdash e : [\tau, \sigma]$.


% \begin{claim}
%   For any term $e$ of Dirac lambda calculus in any context $\Gamma$, there exists at most one type $\tau$ that satisfies $\Gamma \vdash e : \tau$. The types of all terms are computable (if exist).
% \end{claim}


\subsection{Equational Theory}
The following equations are incorporated when considering the equivalence of types and terms.

\subsubsection*{Complex Number}
For example, $\mathrm{add}(\alpha, 0) = \alpha$, $\mathrm{mul}(\alpha, 1) = \alpha$, $\mathrm{mul}(\alpha, \beta) = \mathrm{mul}(\beta, \alpha)$ \dots

\subsubsection*{Type Squash}
\begin{gather*}
  (\tau * \unit)  = \tau \qquad (\unit * \tau) = \tau \\
\end{gather*}
\textbf{Remark:} The type squash is necessary because we have to handle inner product as a scalar. For example, $e \cdot \ket{s}$ where $e$ is a variable of bra shape.

\subsubsection*{Delta Expression}
\begin{align*}
    & \delta_{t, s}.e = \delta_{s, t}.e \\
    & \delta_{a, b}.\delta_{c, d}.e = \delta_{c, d}.\delta_{a, b}.e
\end{align*}

\subsubsection*{Associative function}
\begin{align*}
    A \cdot (B \cdot C) = & A \cdot B \cdot C
\end{align*}

\subsubsection*{AC function}
\begin{align*}
    A + B = & B + A \\
    A + (B + C) = & A + B + C \\
\end{align*}

\yx{Should check whether the equational theory is compatible with the reduction rules.}


\subsection{Simplification Order}

\begin{definition}
  
  We define $<_{RPO}$ as a simplification order on Dirac notations, which is the recursive path order induced by the function order:


  \begin{center}
    \begin{tikzpicture}[node distance=15pt]
      \node[draw]                     (tt) {$\utt$};
      \node[draw, right=of tt]         (var)   {\text{Var}};
      \node[draw, right=of var]       (zero)   {$\mathbf{0}$};
      \node[draw, right=of zero]       (id)   {$\mathbf{1}$};
      \node[draw, right=of id]       (bra) {$\bra{t}$};
      \node[draw, right=of bra]       (ket) {$\ket{t}$};
      \node[draw, right=of ket]      (add)  {$+$};
      \node[draw, right=of add]     (delta)     {$\delta_{ij}$};
      \node[draw, right=of delta]     (scaling)     {$\alpha.e$};
      \node[draw, right=of scaling]     (tensor)     {$\otimes$};
      \node[draw, right=of tensor]       (mul)  {$\cdot$};
      \node[draw, right=of mul]       (trans)     {transpose};
      \node[draw, right=of trans]     (conj)     {conjugate};

      \node[draw, below=of scaling]       (pair) {$(s_1, s_2)$};

      
      \graph{
        (conj) -> (trans) -> (mul) -> (tensor) -> (scaling) -> (delta) -> (add) -> (ket) -> (bra) -> (id) -> (zero) ->(var) -> (tt);
        (tensor) -> (pair);
      };
    \end{tikzpicture}
    \end{center}

\end{definition}

\begin{lemma}
  $<_{RPO}$ is a total order for Dirac notations in any context $\Gamma$.
\end{lemma}

\yx{This property must be guaranteed because it's necessary for the confluence of ordered term rewriting.}


\subsection{Reduction in Dirac lambda calculus}

\subsubsection*{Pair and Projection}
\begin{align*}
  & \textsc{(Squash1)} && \Gamma \vdash (\utt, e) \reduce e 
  && \textsc{(Squash2)} && \Gamma \vdash (e, \utt) \reduce e \\
  & \textsc{(Proj1)} && \Gamma \vdash \fst\ (e_1, e_2) \reduce e_1
  && \textsc{(Proj2)} && \Gamma \vdash \snd\ (e_1, e_2) \reduce e_2  \\
  \\
  & \textsc{(Pair)} && \Gamma \vdash (\fst\ e, \snd\ e)\reduce e
  && \textsc{(Unit)} && \frac{\Gamma \vdash x : \unit \qquad x \neq \utt}{\Gamma \vdash x\reduce\utt} 
\end{align*}

\subsubsection*{Delta Expression}
\begin{align*}
  & \textsc{(DeltaZero)} && 
  \Gamma \vdash \delta_{s, t}.\mathbf{0}_{\tau, \sigma} \reduce \mathbf{0}_{\tau, \sigma}
  \\
  & \textsc{(Delta1)} && 
  \Gamma \vdash \delta_{s, s}.e \reduce e 
  \\
  \\
  & \textcolor{red}{\textsc{(Delta0)}} && 
  \frac{s =? t\ \textrm{has no unifier in}\ \Gamma \qquad \Gamma \vdash e : [\tau, \sigma]}{\Gamma \vdash \delta_{s, t}.e \reduce \textbf{0}_{\tau, \sigma}} \\
  \\
  & \textcolor{red}{\textsc{(DeltaDcp)}} && 
  \frac{\Gamma \vdash u : ( \tau * \sigma ) \qquad \tau \neq \unit \qquad \sigma \neq \unit}{\Gamma \vdash \delta_{u, (s, t)}.e \reduce \delta_{\fst\ u, s}.\delta_{\snd\ u, t}.e} 
  \\
  \\
  & \textsc{(DeltaPair)} &&
  \Gamma \vdash \delta_{\fst\ u, \fst\ v}.\delta_{\snd\ u, \snd\ v}.e \reduce \delta_{u, v}.e 
  \\
  & \textsc{(DeltaDist)} &&
  \Gamma \vdash \delta_{s, t}.(e_1 + e_2) \reduce \delta_{s, t}.e_1 + \delta_{s, t}.e_2
\end{align*}
\textbf{Remark:} Here \textsc{(Delta0)} should check whether the two terms are unifiable on the semantic level, using the same reduction system. In other words, $\delta_{s, t}$ is reduced to 0 only when $s = t$ does not hold for all possible valuations.

\subsubsection*{Ket}
\begin{align*}
  & \textsc{(Ket1)} && \Gamma \vdash \ket{\utt} \reduce \mathbf{1}
  && \textsc{(KetPair)} && \Gamma \vdash \ket{s} \otimes \ket{t}\reduce\ket{(s, t)} 
\end{align*}

\subsubsection*{Bra}
\begin{align*}
  & \textsc{(Bra1)} && \Gamma \vdash \bra{\utt} \reduce \mathbf{1}
  && \textsc{(BraPair)} && \Gamma \vdash \bra{s} \otimes \bra{t}\reduce\bra{(s, t)} 
\end{align*}


\subsubsection*{Conjugate}
\begin{align*}
  & \textsc{(Zero*)} && \Gamma \vdash \textbf{0}_{\tau, \sigma}^* \reduce \textbf{0}_{\tau, \sigma}
  && \textsc{(Id*)} && \Gamma \vdash \mathbf{1}^* \reduce \mathbf{1} \\
  & \textsc{(Delta*)} &&
  \Gamma \vdash (\delta_{s,t}.e)^* \reduce \delta_{s,t}.(e^*)\\
  & \textsc{(Ket*)} && \Gamma \vdash \ket{s}^* \reduce \ket{s}
  && \textsc{(Bra*)} && \Gamma \vdash \bra{s}^* \reduce \bra{s} \\
  & \textsc{(Double*)} && \Gamma \vdash (e^*)^*\reduce e
  && \textsc{(T*)} && \Gamma \vdash (e^T)^* \reduce(e^*)^T \\
  & \textsc{(Scr*)} && \Gamma \vdash (\alpha.e)^* \reduce \mathrm{conj}(\alpha).(e^*) \\
  & \textsc{(Add*)} && \Gamma \vdash (e_1 + e_2)^* \reduce e_1^* + e_2^* 
  && \textsc{(Mul*)} && \Gamma \vdash (e_1 \cdot e_2)^* \reduce e_1^* \cdot e_2^* \\
  & \textsc{(Tsr*)} && \Gamma \vdash (e_1 \otimes e_2)^* \reduce e_1^* \otimes e_2^*
\end{align*}

\subsubsection*{Transpose}
\begin{align*}
  & \textsc{(ZeroT)} && \Gamma \vdash \textbf{0}_{\tau, \sigma}^T \reduce \textbf{0}_{\sigma, \tau}
  && \textsc{(IdT)} && \Gamma \vdash \mathbf{1}^T \reduce \mathbf{1} \\
  & \textsc{(DeltaT)} && \Gamma \vdash (\delta_{s,t}.e)^T \reduce \delta_{s,t}.(e^T) \\
  & \textsc{(KetT)} && \Gamma \vdash \ket{s}^T \reduce \bra{s} 
  && \textsc{(BraT)} && \Gamma \vdash \bra{s}^T \reduce \ket{s} \\
  & \textsc{(DoubleT)} && \Gamma \vdash (e^T)^T \reduce e 
  && \textsc{(ScrT)} && \Gamma \vdash (\alpha.e)^T \reduce \alpha.(e^T) \\
  & \textsc{(AddT)} && \Gamma \vdash (e_1 + e_2)^T \reduce e_1^T + e_2^T
  && \textsc{(MulT)} && \Gamma \vdash (e_1 \cdot e_2)^T \reduce e_2^T \cdot e_1^T \\
  & \textsc{(TsrT)} && \Gamma \vdash (e_1 \otimes e_2)^T \reduce e_1^T \otimes e_2^T
\end{align*}

\subsubsection*{Scaling}
\begin{align*}
  & \textcolor{red}{\textsc{(Scr0)}} && \frac{\Gamma \vdash e : [\tau, \sigma]}{\Gamma \vdash 0.e \reduce \textbf{0}_{\tau, \sigma}}
  && \textsc{(Scr1)} && \Gamma \vdash 1.e \reduce e \\
  \\
  & \textsc{(ScrZero)} && \Gamma \vdash \alpha.\mathbf{0}_{\tau, \sigma} \reduce \mathbf{0}_{\tau, \sigma}
  && \textsc{(ScrDelta)} && \Gamma \vdash \alpha.\delta_{s, t}.e \reduce \delta_{s, t}.\alpha.e \\
  & \textsc{(ScrScr)} && \Gamma \vdash \alpha.\beta.e \reduce \mathrm{mul}(\alpha, \beta).e
  \\
  & \textsc{(ScrDist)} && \Gamma \vdash \alpha.(e_1 + e_2) \reduce \alpha.e_1 + \alpha.e_2
\end{align*}

\subsubsection*{Addition}
\begin{align*}
  & \textsc{(AddZero)} && \Gamma \vdash e + \textbf{0}_{\tau, \sigma} \reduce e \\
  & \textcolor{red}{\textsc{(Fac2)}} && 
    \Gamma \vdash \alpha.e + \beta.e \reduce \mathrm{add}(\alpha, \beta).e
  && \textcolor{red}{\textsc{(Fac1)}} &&
    \Gamma \vdash \alpha.e + e \reduce \mathrm{add}(\alpha, 1).e \\
  & \textcolor{red}{\textsc{(Fac0)}} &&
    \Gamma \vdash e + e \reduce \mathrm{add}(1, 1).e
\end{align*}

\subsubsection*{Multiplication}
\begin{align*}
  & \textcolor{red}{\textsc{(MulZeroL)}}
  && \frac{\Gamma \vdash e : [\tau, \rho]}{\Gamma \vdash \textbf{0}_{\rho, \sigma} \cdot e \reduce \textbf{0}_{\tau, \sigma}}
  && \textcolor{red}{\textsc{(MulZeroR)}}
  && \frac{\Gamma \vdash e : [\rho, \sigma]}{\Gamma \vdash e \cdot \textbf{0}_{\tau, \rho} \reduce \textbf{0}_{\tau, \sigma}} \\
  \\
  & \textsc{(MulIdL)} && \Gamma \vdash \mathbf{1} \cdot e \reduce e 
  && \textsc{(MulIdR)} && \Gamma \vdash e \cdot \mathbf{1} \reduce e 
  \\
  & \textsc{(MulScrL)} && \Gamma \vdash (\alpha.u) \cdot v \reduce \alpha.(u \cdot v)
  && \textsc{(MulScrR)} && \Gamma \vdash u \cdot (\alpha.v) \reduce \alpha.(u \cdot v)
  \\
  & \textsc{(MulDeltaL)} && \Gamma \vdash (\delta_{s, t}.u) \cdot v \reduce \delta_{s, t}.(u \cdot v)
  && \textsc{(MulDeltaR)} && \Gamma \vdash u \cdot (\delta_{s, t}.v) \reduce \delta_{s, t}.(u \cdot v)
  \\
  & \textsc{(MulBraKet)} && \Gamma \vdash \bra{s} \cdot \ket{t} \reduce \delta_{s, t}.\mathbf{1} \\ 
  & \textsc{(MulDistL)} && \Gamma \vdash (u + v) \cdot e \reduce u \cdot e + v \cdot e
  && \textsc{(MulDistR)} && \Gamma \vdash e \cdot (u + v) \reduce e \cdot u + e \cdot v \\
  \\
  & \textsc{(ReFac2)} && \frac{\Gamma \vdash a_1 : [\rho, \tau] \qquad \Gamma \vdash b_1 : [\sigma, \rho]}{\Gamma \vdash (a_1 \otimes a_2) \cdot (b_1 \otimes b_2) \reduce (a_1 \cdot b_1) \otimes (a_2 \cdot b_2)} \\
\end{align*}

\begin{align*}
  & \textsc{(ReFac1LL)} && \frac{\Gamma \vdash e : [\unit, \tau]}{\Gamma \vdash (e \otimes a) \cdot b \reduce e \otimes (a \cdot b)}
  && \textsc{(ReFac1LR)} && \frac{\Gamma \vdash e : [\unit, \tau]}{\Gamma \vdash (a \otimes e) \cdot b \reduce (a \cdot b) \otimes e} \\
  \\
  & \textsc{(ReFac1RL)} && \frac{\Gamma \vdash e : [\tau, \unit]}{\Gamma \vdash a \cdot (e \otimes b) \reduce e \otimes (a \cdot b)} 
  && \textsc{(ReFac1RR)} && \frac{\Gamma \vdash e : [\tau, \unit]}{\Gamma \vdash a \cdot (b \otimes e) \reduce (a \cdot b) \otimes e} \\
  \\
  & \textsc{(ReFac0)} && \frac{\Gamma \vdash a : [\unit, \tau] \qquad \Gamma \vdash b : [\sigma, \unit]}{\Gamma \vdash a \cdot b \reduce a \otimes b}
\end{align*}

\begin{align*}
  & \textsc{(ReFacBPair)} && \frac{\Gamma \vdash u : (\tau * \sigma) \qquad \tau \neq \unit \qquad \sigma \neq \unit \qquad \Gamma \vdash a : [\rho, \tau]}{\Gamma \vdash \bra{u} \cdot (a \otimes b) \reduce (\bra{\fst\ u} \cdot a) \otimes (\bra{\snd\ u} \cdot b)} \\
  \\
  & \textsc{(ReFacKPair)} && \frac{\Gamma \vdash u : (\tau * \sigma) \qquad \tau \neq \unit \qquad \sigma \neq \unit \qquad \Gamma \vdash a : [\sigma, \rho]}{\Gamma \vdash (a \otimes b) \cdot \ket{u}  \reduce (a \cdot \ket{\fst\ u}) \otimes (b \cdot \ket{\snd\ u})}
\end{align*}
% \textbf{Remark: } The last several rules conduct \textbf{tensorization}. They have clear tensor network interpretations. 
% And also, the procedure of tensorization is actually optimizing the contraction order of the tensor network.


\subsubsection*{Tensor}
\begin{align*}
  & \textcolor{red}{\textsc{(TsrZeroL)}} && 
  \frac{\Gamma \vdash e : [\tau, \sigma]}{\Gamma \vdash \textbf{0}_{\tau', \sigma'} \otimes e \reduce \textbf{0}_{(\tau' * \tau), (\sigma' * \sigma)}} 
  && \textcolor{red}{\textsc{(TsrZeroR)}} && 
  \frac{\Gamma \vdash e : [\tau, \sigma]}{\Gamma \vdash e \otimes \textbf{0}_{\tau', \sigma'} \reduce \textbf{0}_{(\tau * \tau'), (\sigma * \sigma')}} \\
  & \textcolor{red}{\textsc{(TsrIdL)}} && 
  \Gamma \vdash \mathbf{1} \otimes e \reduce e
  && \textcolor{red}{\textsc{(TsrIdR)}} && 
  \Gamma \vdash e \otimes \mathbf{1} \reduce e
  \\
  & \textcolor{red}{\textsc{(TsrScrL)}} && \Gamma \vdash (\alpha.u) \otimes v \reduce \alpha.(u \otimes v) 
  && \textcolor{red}{\textsc{(TsrScrR)}} && \Gamma \vdash u \otimes (\alpha.v) \reduce \alpha.(u \otimes v)
  \\
  & \textcolor{red}{\textsc{(TsrDeltaL)}} && \Gamma \vdash (\delta_{s, t}.u) \otimes v \reduce \delta_{s, t}.(u \otimes v) 
  && \textcolor{red}{\textsc{(TsrDeltaR)}} && \Gamma \vdash u \otimes (\delta_{s, t}.v) \reduce \delta_{s, t}.(u \otimes v)
  \\
  & \textsc{(TsrDistL)} && \Gamma \vdash (u + v) \otimes e \reduce u \otimes e + v \otimes e
  && \textsc{(TsrDistR)} && \Gamma \vdash e \otimes (u + v) \reduce e \otimes u + e \otimes v
\end{align*}
  
\begin{align*}
  & \textsc{(TsrComm)} && 
  \frac{
    \begin{aligned}
      & \Gamma \vdash u : [\tau_u, \sigma_u] \qquad \Gamma \vdash v : [\tau_v, \sigma_v] \\
      & \tau_u = \unit \vee \tau_v = \unit \qquad \sigma_u = \unit \vee \sigma_v = \unit
    \end{aligned} 
    \qquad 
    \qquad v <_{RPO} u
    }{\Gamma \vdash u \otimes v \reduce v \otimes u} \\
  \\
  & \textsc{(TsrAssoc)} &&
  \frac{
    \begin{aligned}
      & \Gamma \vdash u : [\tau_u, \sigma_u] \qquad \Gamma \vdash v : [\tau_v, \sigma_v] \qquad \Gamma \vdash w : [\tau_w, \sigma_w] \\
      & \tau_u = \unit \vee \tau_v = \unit \vee \tau_w = \unit \qquad \sigma_u = \unit \vee \sigma_v = \unit \vee \sigma_w = \unit
    \end{aligned}
    }{\Gamma \vdash u \otimes (v \otimes w) \reduce u \otimes v \otimes w} \\
  \\
  & \textsc{(TsrSort1)} &&
  \frac{ 
    \begin{aligned}
      & \Gamma \vdash u : [\tau_u, \sigma_u] \qquad \Gamma \vdash v : [\tau_v, \sigma_v] \qquad \Gamma \vdash w : [\tau_w, \sigma_w] \\
      & \tau_v = \unit \vee \tau_w = \unit \qquad \sigma_v = \unit \vee \sigma_w = \unit
    \end{aligned}
    \qquad
    v <_{RPO} w
  }
  {\Gamma \vdash u \otimes v \otimes w \reduce u \otimes w \otimes v }
  \\
  \\
  & \textsc{(TsrSort2)} &&
  \frac{ 
    \begin{aligned}
      & \Gamma \vdash u : [\tau_u, \sigma_u] \qquad \Gamma \vdash v : [\tau_v, \sigma_v] \qquad \Gamma \vdash w : [\tau_w, \sigma_w] \\
      & \tau_u = \unit \vee (\tau_v = \unit \wedge \tau_w = \unit) \qquad \sigma_u = \unit \vee (\sigma_v = \unit \wedge \sigma_w = \unit)
    \end{aligned}
    \qquad
    u <_{RPO} w
  }
  {\Gamma \vdash u \otimes v \otimes w \reduce v \otimes w \otimes u }
\end{align*}

\yx{The rules in red does not follow the $<_{RPO}$.
\begin{enumerate}
  \item Conflict in the order of function symbols $\otimes$ and $+$, due to the existence of factorization rules and distribution rules.
  \item problem of $\mathbf{0}_{\tau, \sigma}$: the types $\tau$ and $\sigma$ do not appear as subterms in the LHS of the rule.
\end{enumerate}
}

\begin{lemma}[preservation of Dirac types]
  For any context $\Gamma$ and terms $e, e'$, if $\Gamma \vdash e : \tau$ and $\Gamma \vdash e \reduce e'$, we have
  $ \Gamma \vdash e' : \tau $. 
\end{lemma}
\begin{proof}
  By case analysis. It's worth noting that tensoring scalar-like notations will not change the type, becuase $(\unit * \tau) = \tau$ by the equational theory.
\end{proof}

\subsection{Termination and Confluence}

\begin{lemma}[termination]
  The reduction system is terminating. In other words, for any context $\Gamma$ and a well-typed expression $\Gamma \vdash e : \tau$, there does not exists an infinite reduction $\Gamma \vdash e \reduce e' \reduce\cdots .$
\end{lemma}
\begin{proof}
  It is checked that all the rules reduce $<_{RPO}$, with only several exceptions \dots \yx{TO BE PROVED}
\end{proof}


\begin{lemma}[convergence]
  The reduction system is convergent. In other words, for any context $\Gamma$ and a well-typed expression $\Gamma \vdash e : \tau$, there exists a unique normal form for $e$.
\end{lemma}
\begin{proof}
  Since the reduction system is terminating, we only need to check that all the critical pairs are joinable. Here is a thorough list of them:
  \begin{itemize}
    %%%%%%%%%%%%%%%%%%%%%%%%%%%%%%%%%%%%%%%%%%%%%%%%%%%%%%%%%%%%%%%%%%
    \item \textbf{critical pairs containing} \textsc{(Squash1)}/\textsc{(Squash2)} 
    
      \begin{flalign*}
        & (\utt, \utt) \reduce \left \{
          \begin{aligned}
            & \textsc{(Squash1)} && \\
            & \textsc{(Squash2)} &&
          \end{aligned}
        \right \} \reduce \utt &
      \end{flalign*}

      \textbf{Remark:} There is no critical pairs between \textsc{(Squash1)-(DeltaDcp)}: $\delta_{u, (\utt, t)}.e$ does not satisfy the side condition of \textsc{(DeltaDcp)}. Similar for \textsc{(Squash2)-(DeltaDcp)}.

      %%%%%%%%%%%%%%%%%%%%%%%%%%%%%%%%%%%%%%%%%%%%%%%%%%%%%%%%%%%%%%%%%%
      \item \textbf{critical pairs containing} \textsc{(Proj1)}/\textsc{(Proj2)}
      \begin{flalign*}
        & (\fst\ (e_1, e_2), \snd\ (e_1, e_2)) \reduce \left \{
          \begin{aligned}
            & \textsc{(Proj1)} && (e_1, \snd\ (e_1, e_2)) \\
            & \textsc{(Pair)} &&
          \end{aligned}
        \right \} \reduce (e_1, e_2) &
      \end{flalign*}
      \textbf{Remark:} Similar for the \textsc{(Proj2)-(Pair)} pair from $(\fst\ (e_1, e_2), \snd\ (e_1, e_2))$.

      \begin{flalign*}
        & \delta_{\fst\ (a, b), \fst\ u}.\delta_{\snd\ (a, b), \snd\ u}.e \reduce \left \{
          \begin{aligned}
            & \textsc{(Proj1)} && \delta_{a, \fst\ u}.\delta_{\snd\ (a, b), \snd\ u}.e  \\
            & \textsc{(DeltaPair)} && \delta_{(a, b), u}.e 
          \end{aligned}
        \right \} \reduce \delta_{a, \fst\ u}.\delta_{b, \snd\ u}.e &
      \end{flalign*}
      \textbf{Remark:} Similar for the \textsc{(Proj2)-(DeltaPair)} pair from $\delta_{\fst\ (a, b), \fst\ u}.\delta_{\snd\ (a, b), \snd\ u}.e$.


      %%%%%%%%%%%%%%%%%%%%%%%%%%%%%%%%%%%%%%%%%%%%%%%%%%%%%%%%%%%%%%%%%%
      \item \textbf{critical pairs containing} \textsc{(Pair)}
      \begin{flalign*}
        & \delta_{u, (\fst\ v, \snd\ v)}.e \reduce \left \{
          \begin{aligned}
            & \textsc{(Pair)} && \\
            & \textsc{(DeltaDcp)} && \delta_{\fst\ u, \fst\ v}.\delta_{\snd\ u, \snd\ v}.e
          \end{aligned}
        \right \} \reduce \delta_{u, v}.e &
      \end{flalign*}



    %%%%%%%%%%%%%%%%%%%%%%%%%%%%%%%%%%%%%%%%%%%%%%%%%%%%%%%%%%%%%%%%%%
    \item \textbf{critical pairs containing} \textsc{(DeltaZero)}
     
      These critical pairs always join at the zero operator and are all trival.

    %%%%%%%%%%%%%%%%%%%%%%%%%%%%%%%%%%%%%%%%%%%%%%%%%%%%%%%%%%%%%%%%%%
    \item \textbf{critical pairs containing} \textsc{(Delta1)}

      \begin{flalign*}
        & \delta_{(s, t), (s, t)}.e \reduce \left \{
          \begin{aligned}
            & \textsc{(Delta1)} && \\
            & \textsc{(DeltaDcp)} && \delta_{\fst\ (s, t), s}.\delta_{\snd\ (s, t), t}.e \reduce \delta_{s, s}.\delta_{t, t}.e
          \end{aligned}
        \right \} \reduce e &
      \end{flalign*}

      \begin{flalign*}
        & \delta_{\fst\ u, \fst\ u}.\delta_{\snd\ u, \snd\ u}.e \reduce \left \{
          \begin{aligned}
            & \textsc{(Delta1)} && \\
            & \textsc{(DeltaPair)} && \delta_{u, u}.e
          \end{aligned}
        \right \} \reduce e &
      \end{flalign*}

      \begin{flalign*}
        & \delta_{s, s}.(e_1 + e_2) \reduce \left \{
          \begin{aligned}
            & \textsc{(Delta1)} && \\
            & \textsc{(DeltaDist)} && \delta_{s, s}.e_1 + \delta_{s, s}.e_2
          \end{aligned}
        \right \} \reduce e_1 + e_2 &
      \end{flalign*}
      \textbf{Remark:} Other critical pairs are trival:
      \begin{itemize}
        \item \textsc{(Delta1)-(Delta*)} pair from $(\delta_{s, s}.e)^*$,
        \item \textsc{(Delta1)-(DeltaT)} pair from $(\delta_{s, s}.e)^T$,
        \item \textsc{(Delta1)-(ScrDelta)} pair from $\alpha.\delta_{s, s}.e$,
        \item \textsc{(Delta1)-(MulDeltaL)} pair from $(\delta_{s, s}.u) \cdot v$,
        \item \textsc{(Delta1)-(MulDeltaR)} pair from $u \cdot (\delta_{s, s}.v)$,
        \item \textsc{(Delta1)-(TsrDeltaL)} pair from $(\delta_{s, s}.u) \otimes v$, and
        \item \textsc{(Delta1)-(TsrDeltaR)} pair from $u \otimes (\delta_{s, s}.v)$.
      \end{itemize}


    %%%%%%%%%%%%%%%%%%%%%%%%%%%%%%%%%%%%%%%%%%%%%%%%%%%%%%%%%%%%%%%%%%
    \item \textbf{critical pairs containing} \textsc{(Delta0)}

      \begin{flalign*}
        & \delta_{u, (s, t)}.e \reduce \left \{
          \begin{aligned}
            & \textsc{(Delta0)} && \\
            & \textsc{(DeltaDcp)} && \delta_{\fst\ u, s}.\delta_{\snd\ u, t}.e \reduce \cdots
          \end{aligned}
        \right \} \reduce \textbf{0}_{\tau, \sigma} \\
        & \text{($u =? (s, t)$ cannot be unified, $\Gamma \vdash e : [\tau, \sigma]$)} &
      \end{flalign*}
      \textbf{Remark:} If $u =? (s, t)$ cannot be unified, then either $\fst\ u =? s$ or $\snd\ u =? t$ cannot be unified. Therefore the expression will always be reduced to $\textbf{0}$.  
    
      \begin{flalign*}
        & \delta_{\fst\ u, \fst\ v}.\delta_{\snd\ u, \snd\ v}.e \reduce \left \{
          \begin{aligned}
            & \textsc{(Delta0)} && \\
            & \textsc{(DeltaPair)} && \delta_{u, v}.e
          \end{aligned}
        \right \} \reduce \textbf{0}_{\tau, \sigma} \\
        & \text{($\fst\ u =? \fst\ v$ cannot be unified, $\Gamma \vdash e : [\tau, \sigma]$)} &
      \end{flalign*}
      \textbf{Remark:} If $\fst\ u =? \fst\ v$ cannot be unified, then $u =? v$ cannot be unified.

    \textbf{Remark:} Other critical pairs are trival:
    \begin{itemize}
      \item \textsc{(Delta0)-(DeltaDist)} pair from $\delta_{s, t}.(e_1 + e_2)$,
      \item \textsc{(Delta0)-(Delta*)} pair from $(\delta_{s, t}.e)^*$,
      \item \textsc{(Delta0)-(DeltaT)} pair from $(\delta_{s, t}.e)^T$,
      \item \textsc{(Delta0)-(ScrDelta)} pair from $\alpha.\delta_{s, t}.e$,
      \item \textsc{(Delta0)-(MulDeltaL)} pair from $(\delta_{s, t}.u) \cdot v$,
      \item \textsc{(Delta0)-(MulDeltaR)} pair from $u \cdot (\delta_{s, t}.v)$,
      \item \textsc{(Delta0)-(TsrDeltaL)} pair from $(\delta_{s, t}.u) \otimes v$, and
      \item \textsc{(Delta0)-(TsrDeltaR)} pair from $u \otimes (\delta_{s, t}.v)$.
    \end{itemize}

    %%%%%%%%%%%%%%%%%%%%%%%%%%%%%%%%%%%%%%%%%%%%%%%%%%%%%%%%%%%%%%%%%%
    \item \textbf{critical pairs containing} \textsc{(DeltaDcp)}
      \begin{flalign*}
        & \delta_{(a, b), (u, v)}.e \reduce \left \{
          \begin{aligned}
            & \textsc{(DeltaDcp)} && \delta_{\fst\ (a, b), u}.\delta_{\snd\ (a, b), v}.e \\
            & \textsc{(DeltaDcp)} && \delta_{a, \fst\ (u, v)}.\delta_{b, \snd\ (u, v)}.e
          \end{aligned}
        \right \} \reduce \delta_{a, u}.\delta_{b, v}.e &
      \end{flalign*}

      \textbf{Remark:} Other critical pairs are trival.

    %%%%%%%%%%%%%%%%%%%%%%%%%%%%%%%%%%%%%%%%%%%%%%%%%%%%%%%%%%%%%%%%%%
    \item \textbf{critical pairs containing} \textsc{(DeltaPair)}
    
    No extra critical pairs.
    % \begin{flalign*}
    %   & \delta_{\fst\ u, \fst\ v}.\delta_{\snd\ u, \snd\ v}.\delta_{\fst\ u, \fst\ v}.e \reduce \left \{
    %     \begin{aligned}
    %       & \textsc{(DeltaPair)} && \delta_{u, v}.\delta_{\fst\ u, \fst\ v}.e \\
    %       & \textsc{(DeltaPair)} && \delta_{\fst\ u, \fst\ v}.\delta_{u, v}.e
    %     \end{aligned}
    %   \right \} \reduce \delta_{u, v}.\delta_{\fst\ u, \fst\ v}.e &
    % \end{flalign*}
    % \textbf{Remark:} Similar for another critical pair from $\delta_{\fst\ u, \fst\ v}.\delta_{\snd\ u, \snd\ v}.\delta_{\snd\ u, \snd\ v}.e$.
    
    %%%%%%%%%%%%%%%%%%%%%%%%%%%%%%%%%%%%%%%%%%%%%%%%%%%%%%%%%%%%%%%%%%
    \item \textbf{critical pairs containing} \textsc{(DeltaDist)}
    
    All critical pairs are trivial.
  
    %%%%%%%%%%%%%%%%%%%%%%%%%%%%%%%%%%%%%%%%%%%%%%%%%%%%%%%%%%%%%%%%
    \item \textbf{critical pairs containing} \textsc{(Ket1)}/\textsc{(Bra1)}
    
    Here we present all the critical pairs involves ket $\ket{\utt}$. Those for bra $\bra{\utt}$ are totally symmetric.

      \begin{flalign*}
        & \ket{\utt} \otimes \ket{s} \reduce \left \{
          \begin{aligned}
            & \textsc{(Ket1)} && \mathbf{1} \otimes \ket{s} \\
            & \textsc{(KetPair)} && \ket{(tt, s)}
          \end{aligned}
        \right \} \reduce \ket{s} &
      \end{flalign*}
      \textbf{Remark:} Similar for the \textsc{(Ket1)-(KetPair)} from $\ket{s} \otimes \ket{\utt}$.

      \begin{flalign*}
        & \ket{\utt}^* \reduce \left \{
          \begin{aligned}
            & \textsc{(Ket1)} && \mathbf{1}^* \\
            & \textsc{(Ket*)} && \ket{\utt}
          \end{aligned}
        \right \} \reduce \mathbf{1} &
      \end{flalign*}

      \begin{flalign*}
        & \ket{\utt}^T \reduce \left \{
          \begin{aligned}
            & \textsc{(Ket1)} && \mathbf{1}^T \\
            & \textsc{(KetT)} && \bra{\utt}
          \end{aligned}
        \right \} \reduce \mathbf{1} &
      \end{flalign*}

      \begin{flalign*}
        & \bra{s} \cdot \ket{\utt} \reduce \left \{
          \begin{aligned}
            & \textsc{(Ket1)}&& \bra{s} \cdot \mathbf{1} \reduce \bra{\utt} \cdot \mathbf{1} \reduce \mathbf{1} \cdot \mathbf{1} \\
            & \textsc{(MulBraKet)} && \delta_{s, \utt}.\mathbf{1} \reduce \delta_{\utt, \utt}.\mathbf{1}
          \end{aligned}
        \right \} \reduce \mathbf{1} &
      \end{flalign*}
      \textbf{Remark:} $\Gamma \vdash s : \unit$ holds in this case. 

  %%%%%%%%%%%%%%%%%%%%%%%%%%%%%%%%%%%%%%%%%%%%%%%%%%%%%%%%%%%%%%%%
  \item \textbf{critical pairs containing} \textsc{(KetPair)}/\textsc{(BraPair)}
  
  Here we present all the critical pairs involves ket $\ket{s} \otimes \ket{t}$. Those for bra $\bra{s} \otimes \bra{t}$ are totally symmetric.

  \begin{flalign*}
      & (\ket{s} \otimes \ket{t})^* \reduce \left \{
        \begin{aligned}
          & \textsc{(KetPair)} && \ket{(s, t)}^* \\
          & \textsc{(Tsr*)} && \ket{s}^* \otimes \ket{t}^* \reduce \ket{s} \otimes \ket{t}
        \end{aligned}
      \right \} \reduce \ket{(s, t)} &
    \end{flalign*}

    \begin{flalign*}
      & (\ket{s} \otimes \ket{t})^T \reduce \left \{
        \begin{aligned}
          & \textsc{(KetPair)} && \ket{(s, t)}^T \\
          & \textsc{(TsrT)} && \ket{s}^T \otimes \ket{t}^T \reduce \bra{s} \otimes \bra{t}
        \end{aligned}
      \right \} \reduce \bra{(s, t)} &
    \end{flalign*}

    \begin{flalign*}
      & (a \otimes b) \cdot (\ket{s} \otimes \ket{t}) \reduce \left \{
        \begin{aligned}
          & \textsc{(KetPair)} && (a \otimes b) \cdot \ket{(s, t)} \\
          & \textsc{(ReFac2)} && 
        \end{aligned}
      \right \} \reduce (a \cdot \ket{s}) \otimes (b \cdot \ket{t}) \\
      & (\Gamma \vdash a : [\rho, \tau], \Gamma \vdash \ket{s} : [\unit, \rho]) &
    \end{flalign*}

    %%%%%%%%%%%%%%%%%%%%%%%%%%%%%%%%%%%%%%%%%%%%%%%%%%%%%%%%%%%%%%%%
    \item \textbf{conjugate critical pairs}    
    
      \begin{flalign*}
      & (\mathbf{0}_{\tau, \sigma}^*)^* \reduce \left \{
        \begin{aligned}
          & \textsc{(Double*)} && \\
          & \textsc{(Zero*)} && (\mathbf{0}_{\tau, \sigma})^* 
        \end{aligned}
        \right \} \reduce \mathbf{0}_{\tau, \sigma}&
      \end{flalign*}
      \textbf{Remark:} Similar for $((\mathbf{1})^*)^*$, $((\delta_{s, t}.e)^*)^*$, $(\ket{s}^*)^*$ and $(\bra{s}^*)^*$.
      
      \begin{flalign*}
        & ((\alpha.e)^*)^* \reduce \left \{
          \begin{aligned}
            & \textsc{(Double*)} && \\
            & \textsc{(C*)} && (\mathrm{conj}(\alpha).e^*)^* \reduce \mathrm{conj}(\mathrm{conj}(\alpha)).e 
          \end{aligned}
        \right \} \reduce \alpha.e &
      \end{flalign*}

      \begin{flalign*}
        & ((e^*)^*)^* \reduce \left \{
          \begin{aligned}
            & \textsc{(Double*)} && \\
            & \textsc{(Double*)} &&
          \end{aligned}
        \right \} \reduce e^* &
      \end{flalign*}
      \textbf{Remark:} The \textsc{(Double*)} rule can overlap with itself.

      \begin{flalign*}
        & ((e^T)^*)^* \reduce \left \{
          \begin{aligned}
            & \textsc{(Double*)} && \\
            & \textsc{(T*)} && ((e^*)^T)^* \reduce ((e^*)^*)^T 
          \end{aligned}
        \right \} \reduce e^T &
      \end{flalign*}

      \begin{flalign*}
        & ((e_1 + e_2)^*)^* \reduce \left \{
          \begin{aligned}
            & \textsc{(Double*)} && \\
            & \textsc{(Add*)} && (e_1^* + e_2^*)^* \reduce (e_1^*)^* + (e_2^*)^*
          \end{aligned}
        \right \} \reduce e_1 + e_2 &
      \end{flalign*}
      \textbf{Remark:} Similar for $((e_1 \cdot e_2)^*)^*$, $((e_1 \otimes e_2)^*)^*$.

    %%%%%%%%%%%%%%%%%%%%%%%%%%%%%%%%%%%%%%%%%%%%%%%%%%%%%%%%%%%%%%%%
    \item \textbf{transpose critical pairs}
    
      The critical pairs related to transpose reduction rules are similar to those of conjugate.

    %%%%%%%%%%%%%%%%%%%%%%%%%%%%%%%%%%%%%%%%%%%%%%%%%%%%%%%%%%%%%%%%
    \item \textbf{critical pairs containing} \textsc{(Scr0)}
    
      \begin{flalign*}
        & 0.\mathbf{0}_{\tau, \sigma} \reduce \left \{
          \begin{aligned}
            & \textsc{(Scr0)} && \\
            & \textsc{(ScrZero)} && 
          \end{aligned}
        \right \} \reduce \mathbf{0}_{\tau, \sigma} &
      \end{flalign*}

      \begin{flalign*}
        & 0.\delta_{s, t}.e \reduce \left \{
          \begin{aligned}
            & \textsc{(Scr0)} && \\
            & \textsc{(ScrDelta)} && \delta_{s, t}.0.e \reduce \delta_{s, t}.\mathbf{0}_{\tau, \sigma}
          \end{aligned}
        \right \} \reduce \mathbf{0}_{\tau, \sigma} \qquad (\Gamma \vdash e : [\tau, \sigma]) &
      \end{flalign*}

      \begin{flalign*}
        & 0.\alpha.e \reduce \left \{
          \begin{aligned}
            & \textsc{(Scr0)} && \\
            & \textsc{(ScrScr)} && \mathrm{mul}(0, \alpha).e
          \end{aligned}
        \right \} \reduce \mathbf{0}_{\tau, \sigma} \qquad (\Gamma \vdash e : [\tau, \sigma]) &
      \end{flalign*}

      \begin{flalign*}
        & 0.(e_1 + e_2) \reduce \left \{
          \begin{aligned}
            & \textsc{(Scr0)} && \\
            & \textsc{(ScrDist)} && 0.e_1 + 0.e_2 \reduce \mathbf{0}_{\tau, \sigma} + 0.e_2 \reduce \mathbf{0}_{\tau, \sigma} + \mathbf{0}_{\tau, \sigma}
          \end{aligned}
        \right \} \reduce \mathbf{0}_{\tau, \sigma} \qquad (\Gamma \vdash (e_1 + e_2) : [\tau, \sigma]) &
      \end{flalign*}

      \begin{flalign*}
        & 0.e + \beta.e \reduce \left \{
          \begin{aligned}
            & \textsc{(Scr0)} && \mathbf{0}_{\tau, \sigma} + \beta.e \\
            & \textsc{(Fac2)} && \mathrm{add}(0, \beta).e
          \end{aligned}
        \right \} \reduce \beta.e \qquad (\Gamma \vdash e : [\tau, \sigma]) &
      \end{flalign*}
      \textbf{Remark:} Similar for the \textsc{(Scr0)-(Fac1)} pair from $0.e + e$

      \begin{flalign*}
        & (0.u) \cdot v \reduce \left \{
          \begin{aligned}
            & \textsc{(Scr0)} && \mathbf{0}_{\tau, \sigma} \cdot v \\
            & \textsc{(MulScrL)} && 0.(u \cdot v)
          \end{aligned}
        \right \} \reduce \mathbf{0}_{\rho, \sigma} \qquad (\Gamma \vdash u : [\tau, \sigma], \Gamma \vdash v : [\rho, \tau]) &
      \end{flalign*}
      \textbf{Remark:} Similar for the \textsc{(Scr0)-(MulScrR)} pair from $u \cdot (0.v)$, the \textsc{(Scr0)-(TsrScrL)} pair from $(0.u) \otimes v$ and the \textsc{(Scr0)-(TsrScrR)} pair from $u \otimes (0.v)$.

    %%%%%%%%%%%%%%%%%%%%%%%%%%%%%%%%%%%%%%%%%%%%%%%%%%%%%%%%%%%%%%%%
    \item \textbf{critical pairs containing} \textsc{(Scr1)}
      
      \begin{flalign*}
        & 1.\mathbf{0}_{\tau, \sigma} \reduce \left \{
          \begin{aligned}
            & \textsc{(Scr1)} && \\
            & \textsc{(ScrZero)} && 
          \end{aligned}
        \right \} \reduce \mathbf{0}_{\tau, \sigma} &
      \end{flalign*}


      \begin{flalign*}
        & 1.\delta_{s, t}.e \reduce \left \{
          \begin{aligned}
            & \textsc{(Scr1)} && \\
            & \textsc{(ScrDelta)} && \delta_{s, t}.1.e
          \end{aligned}
        \right \} \reduce \delta_{s, t}.e &
      \end{flalign*}

      \begin{flalign*}
        & 1.\alpha.e \reduce \left \{
          \begin{aligned}
            & \textsc{(Scr1)} && \\
            & \textsc{(ScrScr)} && \mathrm{mul}(1, \alpha).e
          \end{aligned}
        \right \} \reduce \alpha.e &
      \end{flalign*}

      \begin{flalign*}
        & 1.(e_1 + e_2) \reduce \left \{
          \begin{aligned}
            & \textsc{(Scr1)} && \\
            & \textsc{(ScrDist)} && 1.e_1 + 1.e_2 \reduce e_1 + 1.e_2
          \end{aligned}
        \right \} \reduce e_1 + e_2 &
      \end{flalign*}

      \begin{flalign*}
        & 1.e + \beta.e \reduce \left \{
          \begin{aligned}
            & \textsc{(Scr1)} && e + \beta.e \\
            & \textsc{(Fac2)} && 
          \end{aligned}
        \right \} \reduce \mathrm{add}(1, \beta).e &
      \end{flalign*}
      \textbf{Remark:} Similar for the \textsc{(Scr1)-(Fac1)} pair from $1.e + e$.

      \begin{flalign*}
        & (1.u) \cdot v \reduce \left \{
          \begin{aligned}
            & \textsc{(Scr1)} && \\
            & \textsc{(MulScrL)} && 1.(u \cdot v)
          \end{aligned}
        \right \} \reduce u \cdot v &
      \end{flalign*}
      \textbf{Remark:} Similar for the \textsc{(Scr1)-(MulScrR)} pair from $u \cdot (1.v)$, the \textsc{(Scr1)-(TsrScrL)} pair from $(1.u) \otimes v$ and the \textsc{(Scr1)-(TsrScrR)} pair from $u \otimes (1.v)$.

    %%%%%%%%%%%%%%%%%%%%%%%%%%%%%%%%%%%%%%%%%%%%%%%%%%%%%%%%%%%%%%%%
    \item \textbf{critical pairs containing} \textsc{(ScrZero)}

      \begin{flalign*}
        & \alpha.\mathbf{0}_{\tau, \sigma} + \beta.\mathbf{0}_{\tau, \sigma} \reduce \left \{
          \begin{aligned}
            & \textsc{(ScrZero)} && \mathbf{0}_{\tau, \sigma} + \beta.\mathbf{0}_{\tau, \sigma} \reduce \mathbf{0}_{\tau, \sigma} + \mathbf{0}_{\tau, \sigma} \\
            & \textsc{(Fac2)} && \mathrm{add}(\alpha, \beta). \mathbf{0}_{\tau, \sigma} 
          \end{aligned}
          \right \} \reduce \mathbf{0}_{\tau, \sigma} &
        \end{flalign*}
        \textbf{Remark:} Similar for the \textsc{(ScrZero)-(Fac1)} pair from $\alpha.\textbf{0}_{\tau, \sigma} + \textbf{0}_{\tau, \sigma}$.
          
        \begin{flalign*}
          & (\alpha.\textbf{0}_{\tau, \sigma}) \cdot v \reduce \left \{
            \begin{aligned}
              & \textsc{(ScrZero)} && \textbf{0}_{\tau, \sigma} \cdot v\\
              & \textsc{(MulScrL)} && \alpha.(\textbf{0}_{\tau, \sigma} \cdot v)
            \end{aligned}
          \right \} \reduce \textbf{0}_{\rho, \sigma} \qquad (\Gamma \vdash v : [\rho, \tau])&
        \end{flalign*}
        \textbf{Remark:} Similar for the \textsc{(ScrZero)-(MulScrR)} pair from $u \cdot (\alpha.\textbf{0}_{\tau, \sigma})$, the \textsc{(ScrZero)-(TsrScrL)} pair from $(\alpha.\textbf{0}_{\tau, \sigma}) \otimes v$ and the \textsc{(ScrZero)-(TsrScrR)} pair from $u \otimes (\alpha.\textbf{0}_{\tau, \sigma})$.
  

    %%%%%%%%%%%%%%%%%%%%%%%%%%%%%%%%%%%%%%%%%%%%%%%%%%%%%%%%%%%%%%%%
    \item \textbf{critical pairs containing} \textsc{(AddZero)}
    
      \begin{flalign*}
        & \mathbf{0}_{\tau, \sigma} + \mathbf{0}_{\tau, \sigma} \reduce 
        \left \{
          \begin{aligned}
            & \textsc{(AddZero)} && \\
            & \textsc{(AddZero)} && 
          \end{aligned}
          \right \} \reduce \mathbf{0}_{\tau, \sigma} &
      \end{flalign*}
      \textbf{Remark:} The \textsc{(AddZero)} can overlap with itself.


        \begin{flalign*}
          & \alpha.\mathbf{0}_{\tau, \sigma} + \mathbf{0}_{\tau, \sigma} \reduce \left \{
            \begin{aligned}
              & \textsc{(AddZero)} && \alpha.\mathbf{0}_{\tau, \sigma} \\
              & \textsc{(Fac1)} && \mathrm{add}(\alpha, 1). \mathbf{0}_{\tau, \sigma} 
            \end{aligned}
            \right \} \reduce \mathbf{0}_{\tau, \sigma} &
          \end{flalign*}

      \begin{flalign*}
        & \mathbf{0}_{\tau, \sigma} + \mathbf{0}_{\tau, \sigma} \reduce 
        \left \{
          \begin{aligned}
            & \textsc{(AddZero)} && \\
            & \textsc{(Fac0)} && \mathrm{add}(1, 1). \mathbf{0}_{\tau, \sigma}
          \end{aligned}
          \right \} \reduce \mathbf{0}_{\tau, \sigma} &
      \end{flalign*}     

        \begin{flalign*}
          & e \cdot (u + \mathbf{0}_{\tau, \sigma}) \reduce \left \{
            \begin{aligned}
              & \textsc{(AddZero)} && \\
              & \textsc{(MulDistR)} && e \cdot u + e \cdot \mathbf{0}_{\tau, \sigma} \reduce e \cdot u + \mathbf{0}_{\tau, \rho}
            \end{aligned}
          \right \} \reduce e \cdot u \qquad (\Gamma \vdash e : [\sigma, \rho])&
        \end{flalign*}
        \textbf{Remark:} Similar for $(u + \mathbf{0}_{\tau, \sigma}) \cdot e$, $e \otimes (u + \mathbf{0}_{\tau, \sigma})$ and $(u + \mathbf{0}_{\tau, \sigma}) \otimes e$.


    %%%%%%%%%%%%%%%%%%%%%%%%%%%%%%%%%%%%%%%%%%%%%%%%%%%%%%%%%%%%%%%%
    \item \textbf{critical pairs containing} \textsc{(Fac2)}/\textsc{(Fac1)}/\textsc{(Fac0)}

      \begin{flalign*}
        & \alpha.e + \alpha.e \reduce 
        \left \{
          \begin{aligned}
            & \textsc{(Fac2)} && \\
            & \textsc{(Fac0)} && \mathrm{add}(1, 1). \alpha. e \reduce \mathrm{mul}(\mathrm{add}(1, 1), \alpha).e
          \end{aligned}
          \right \} \reduce \mathrm{add}(\alpha, \alpha).e &
      \end{flalign*}     
        
        \begin{flalign*}
          & e \cdot (\alpha.u + \beta.u) \reduce \left \{
            \begin{aligned}
              & \textsc{(Fac2)} && e \cdot (\mathrm{add}(\alpha, \beta).u)\\
              & \textsc{(MulDistR)} && e \cdot (\alpha.u) + e \cdot (\beta.u) \reduce \cdots \reduce \alpha.(e \cdot u) + \beta.(e \cdot u)
            \end{aligned}
          \right \} \reduce \mathrm{add}(\alpha, \beta).(e \cdot u) &
        \end{flalign*}
        \textbf{Remark:} Similar for $(\alpha.u + \beta.u) \cdot e$, $e \otimes (\alpha.u + \beta.u)$ and $(\alpha.u + \beta.u) \otimes e$. Also similar for the critical pairs between \textsc{(Fac1)}, \textsc{(Fac0)} and \textsc{(MulDistL(R))}, \textsc{(TsrDistL(R))}.

        %%%%%%%%%%%%%%%%%%%%%%%%%%%%%%%%%%%%%%%%%%%%%%%%%%%%%%%%%%%%%%%%
        \item \textbf{critical pairs containing} \textsc{(MulZeroL)}/\textsc{(MulZeroR)}
        
          \begin{flalign*}
            & \mathbf{0}_{\rho, \sigma} \cdot \mathbf{0}_{\tau, \rho} \reduce \left \{
                \begin{aligned}
                  & \textsc{(MulZeroL)} \\
                  & \textsc{(MulZeroR)}
                \end{aligned}
                \right \} \reduce \mathbf{0}_{\tau, \sigma} &
          \end{flalign*}

          \begin{flalign*}
            & \mathbf{0}_{\tau, \sigma} \cdot \mathbf{1} \reduce \left \{
                \begin{aligned}
                  & \textsc{(MulZeroL)} \\
                  & \textsc{(MulIdR)}
                \end{aligned}
                \right \} \reduce \mathbf{0}_{\tau, \sigma} &
          \end{flalign*}
          \textbf{Remark:} Similar for $\mathbf{1} \cdot \mathbf{0}_{\tau, \sigma}$.

          \begin{flalign*}
            & (\alpha.e) \cdot \mathbf{0}_{\tau, \sigma} \reduce \left \{
                \begin{aligned}
                  & \textsc{(MulZeroR)} && \\
                  & \textsc{(MulScrL)} && \alpha.(e \cdot \mathbf{0}_{\tau, \sigma})
                \end{aligned}
                \right \} \reduce \mathbf{0}_{\tau, \rho} \qquad (\Gamma \vdash e : [\sigma, \rho]) &
          \end{flalign*}
          \textbf{Remark:} Similar for $\mathbf{0}_{\tau, \sigma} \cdot (\alpha.e)$, $(\delta_{s, t}.e)\cdot \mathbf{0}_{\tau, \sigma}$ and $\mathbf{0}_{\tau, \sigma} \cdot (\delta_{s, t}.e)$.

          \begin{flalign*}
            & \mathbf{0}_{\rho, \sigma} \cdot (e_1 + e_2) \reduce \left \{
              \begin{aligned}
                & \textsc{(MulZeroL)} && \\
                & \textsc{(MulDistR)} && \mathbf{0}_{\rho, \sigma} \cdot e_1 + \mathbf{0}_{\rho, \sigma} \cdot e_2 \reduce \mathbf{0}_{\tau, \sigma} + \mathbf{0}_{\tau, \sigma} 
              \end{aligned}
              \right \} \reduce \mathbf{0}_{\tau, \sigma} \qquad (\Gamma \vdash (e_1 + e_2) : [\tau, \rho])&
          \end{flalign*}
          \textbf{Remark:} Similar for $(e_1 + e_2) \cdot \mathbf{0}_{\tau, \rho}$
          
          \begin{flalign*}
            & \mathbf{0}_{\tau, \sigma} \cdot (e \otimes b) \reduce \left \{
              \begin{aligned}
                & \textsc{(MulZeroL)} && \\
                & \textsc{(ReFac1RL)} && e \otimes (\textbf{0}_{\tau, \sigma} \cdot b) \reduce e \otimes \mathbf{0}_{\rho, \sigma}
              \end{aligned}
              \right \} \reduce \mathbf{0}_{(\eta * \rho), \sigma} \qquad (\Gamma \vdash e : [\eta, \unit], \Gamma \vdash b : [\rho, \tau])&
          \end{flalign*}
          \textbf{Remark:} Similar for the critical pairs between \textsc{(MulZeroL)-(ReFac1RR)}, \textsc{(MulZeroR)-(ReFac1LL)} and \textsc{(MulZeroR)-(ReFac1LR)}.

          \begin{flalign*}
            & \mathbf{0}_{\unit, \sigma} \cdot e \reduce \left \{
              \begin{aligned}
                & \textsc{(MulZeroL)} && \\
                & \textsc{(ReFac0)} && \mathbf{0}_{\unit, \sigma} \otimes e
              \end{aligned}
              \right \} \reduce \mathbf{0}_{\tau, \sigma} \qquad (\Gamma \vdash e : [\tau, \unit]) &
          \end{flalign*}
          \textbf{Remark:} Similar for $e \cdot \mathbf{0}_{\tau, \unit}$.

        %%%%%%%%%%%%%%%%%%%%%%%%%%%%%%%%%%%%%%%%%%%%%%%%%%%%%%%%%%%%%%%%
        \item \textbf{critical pairs containing} \textsc{(MulIdL)}/\textsc{(MulIdR)}
        
        All critical pairs are trivial.

        %%%%%%%%%%%%%%%%%%%%%%%%%%%%%%%%%%%%%%%%%%%%%%%%%%%%%%%%%%%%%%%%
        \item \textbf{critical pairs containing} \textsc{(MulScrL)}/\textsc{(MulScrR)}/\textsc{(MulDeltaL)}/\textsc{(MulDeltaR)}
        
        All critical pairs are trivial.
        
        %%%%%%%%%%%%%%%%%%%%%%%%%%%%%%%%%%%%%%%%%%%%%%%%%%%%%%%%%%%%%%%%
        \item \textbf{critical pairs containing} \textsc{(MulBraKet)}

        \begin{flalign*}
            & \bra{s} \cdot \ket{t} \reduce \left \{
              \begin{aligned}
                & \textsc{(MulBraKet)} && \delta_{s, t}.\mathbf{1} \reduce \delta_{\utt, \utt}.\mathbf{1} \\
                & \textsc{(ReFac0)} && \bra{s} \otimes \ket{t} \reduce \bra{\utt} \otimes \ket{\utt} \reduce \cdots
              \end{aligned}
              \right \} \reduce \mathbf{1}
              \qquad (\Gamma \vdash s : \unit, \Gamma \vdash t : \unit) &
          \end{flalign*}


        %%%%%%%%%%%%%%%%%%%%%%%%%%%%%%%%%%%%%%%%%%%%%%%%%%%%%%%%%%%%%%%%
        \item \textbf{critical pairs containing} \textsc{(MulDistL)}/\textsc{(MulDistR)}

          \begin{flalign*}
            & (a_1 + a_2) \cdot (b_1 + b_2) \reduce \left \{
              \begin{aligned}
                & \textsc{(MulDistR)} && (a_1 + a_2) \cdot b_1 + (a_1 + a_2) \cdot b_2  \\
                & \textsc{(MulDistL)} && a_1 \cdot (b_1 + b_2) + a_2 \cdot (b_1 + b_2)
              \end{aligned}
            \right \} \\
            & \qquad \qquad \reduce a_1 \cdot b_1 + a_1 \cdot b_2 + a_2 \cdot b_1 + a_2 \cdot b_2 &
          \end{flalign*}

          \begin{flalign*}
            & e \cdot (u + v) \reduce \left \{
              \begin{aligned}
                & \textsc{(MulDistR)} && e \cdot u + e \cdot v  \\
                & \textsc{(ReFac0)} && e \otimes (u + v) 
              \end{aligned}
            \right \} \reduce e \otimes u + e \otimes v \qquad (\Gamma \vdash e : [\tau, \unit], \Gamma \vdash (u + v) : [\unit, \sigma]) &
          \end{flalign*}
          \textbf{Remark:} Similar for $(u + v) \cdot e$.


        %%%%%%%%%%%%%%%%%%%%%%%%%%%%%%%%%%%%%%%%%%%%%%%%%%%%%%%%%%%%%%%%
        \item \textbf{critical pairs containing} \textsc{(ReFac2)}/\textsc{(ReFac1)}/\textsc{(ReFac0)}

          \begin{flalign*}
            & (a_1 \otimes a_2) \cdot (b_1 \otimes b_2) \reduce \left \{
              \begin{aligned}
                & \textsc{(ReFac2)} && (a_1 \cdot b_1) \otimes (a_2 \cdot b_2) \\
                & && \reduce (a_1 \otimes b_1) \otimes (a_2 \otimes b_2) \\
                & && \reduce ((a_1 \otimes b_1) \otimes a_2) \otimes b_2 \\
                & \textsc{(ReFac0)} && (a_1 \otimes a_2) \otimes (b_1 \otimes b_2) \\
                & && \reduce ((a_1 \otimes a_2) \otimes b_1) \otimes b_2
              \end{aligned}
            \right \} \reduce 
            \left \{
              \begin{aligned}
                & ((a_1 \otimes b_1) \otimes a_2) \otimes b_2 & (b_1 <_{RPO} a_2) \\
                & ((a_1 \otimes a_2) \otimes b_1) \otimes b_2 & (a_2 <_{RPO} b_2)
              \end{aligned}
            \right . \\
            & (\Gamma \vdash a_1 : [\unit, \tau_1], \Gamma \vdash a_2 : [\unit, \tau_2], \Gamma \vdash b_1 : [\sigma_1, \unit], \Gamma \vdash b_2 : [\sigma_2, \unit]) &
          \end{flalign*}

          \textbf{Remark:} Similar for other critical pairs among \textsc{(ReFac2)}, \textsc{(ReFac1)} and \textsc{(ReFac0)}.

        %%%%%%%%%%%%%%%%%%%%%%%%%%%%%%%%%%%%%%%%%%%%%%%%%%%%%%%%%%%%%%%%
        \item \textbf{critical pairs containing} \textsc{(TsrZeroL)}/\textsc{(TsrZeroR)}
        
          \begin{flalign*}
            & \mathbf{0}_{\tau, \sigma} \otimes \mathbf{0}_{\tau', \sigma'}\reduce \left \{
              \begin{aligned}
                & \textsc{(TsrZeroL)} && \\
                & \textsc{(TsrZeroR)} &&
              \end{aligned}
              \right \} \reduce \mathbf{0}_{(\tau * \tau'), (\sigma * \sigma')} & 
          \end{flalign*}

          \begin{flalign*}
            & \mathbf{0}_{\tau, \sigma} \otimes \mathbf{1} \reduce \left \{
              \begin{aligned}
                & \textsc{(TsrZeroL)} && \\
                & \textsc{(TsrIdR)} &&
              \end{aligned}
              \right \} \reduce \mathbf{0}_{\tau, \sigma} & 
          \end{flalign*}
          \textbf{Remark:} Similar for $\mathbf{1} \otimes \mathbf{0}_{\tau, \sigma}$.

          \begin{flalign*}
            & (\alpha.u) \otimes \mathbf{0}_{\tau', \sigma'} \reduce \left \{
              \begin{aligned}
                & \textsc{(TsrZeroR)} && \\
                & \textsc{(TsrScrL)} && \alpha.(u \otimes \mathbf{0}_{\tau', \sigma'})
              \end{aligned}
              \right \} \reduce \mathbf{0}_{(\tau * \tau'), (\sigma * \sigma')} \qquad (\Gamma \vdash u : [\tau, \sigma]) & 
          \end{flalign*}
          \textbf{Remark:} Similar for $\mathbf{0}_{\tau, \sigma}\otimes (\alpha.v)$, $(\delta_{s, t}.u) \otimes \mathbf{0}_{\tau', \sigma'}$ and $\mathbf{0}_{\tau, \sigma} \otimes (\delta_{s, t}.v)$.


          \begin{flalign*}
            & \mathbf{0}_{\tau, \sigma} \otimes (u + v) \reduce \left \{
              \begin{aligned}
                & \textsc{(TsrZeroL)} && \\
                & \textsc{(TsrDistR)} && \mathbf{0}_{\tau, \sigma} \otimes u + \mathbf{0}_{\tau, \sigma} \otimes v \reduce \mathbf{0}_{(\tau * \tau'), (\sigma * \sigma')} + \mathbf{0}_{(\tau * \tau'), (\sigma * \sigma')}
              \end{aligned}
              \right \} \reduce \mathbf{0}_{(\tau * \tau'), (\sigma * \sigma')} \\
              & (\Gamma \vdash (u + v) : [\tau', \sigma']) & 
          \end{flalign*}
          \textbf{Remark:} Similar for $(u + v) \otimes \mathbf{0}_{\tau, \sigma}$.


          \begin{flalign*}
            & \mathbf{0}_{\tau, \sigma} \otimes e \reduce \left \{
              \begin{aligned}
                & \textsc{(TsrZeroL)} && \\
                & \textsc{(TsrComm)} && e \otimes \mathbf{0}_{\tau, \sigma}
              \end{aligned}
              \right \} \reduce \mathbf{0} \qquad (\Gamma \vdash e : [\unit, \sigma]) & 
          \end{flalign*}
          \textbf{Remark:} Similar for $e \otimes \mathbf{0}_{\unit, \sigma}$. Also similar for critical pairs between \textsc{(TsrZeroL)}, \textsc{(TsrZeroR)} and \textsc{(TsrAssoc)}, \textsc{(TsrSort)}.


        %%%%%%%%%%%%%%%%%%%%%%%%%%%%%%%%%%%%%%%%%%%%%%%%%%%%%%%%%%%%%%%%
        \item \textbf{critical pairs containing} \textsc{(TsrIdL)}/\textsc{(TsrIdR)}
        
        \begin{flalign*}
          & \mathbf{1} \otimes \mathbf{1} \reduce \left \{
            \begin{aligned}
              & \textsc{(TsrIdL)} && \\
              & \textsc{(TsrIdR)} &&
            \end{aligned}
            \right \} \mathbf{1} & 
        \end{flalign*}

        
        \begin{flalign*}
          & \mathbf{1} \otimes \alpha.e \reduce \left \{
            \begin{aligned}
              & \textsc{(TsrIdL)} && \\
              & \textsc{(TsrScrR)} && \alpha.(\mathbf{1} \otimes e)
            \end{aligned}
            \right \} \reduce \alpha.e & 
        \end{flalign*}
        \textbf{Remark:} Similar for $\alpha.e \otimes \mathbf{1}$, $\mathbf{1} \otimes (\delta_{s, t}.e)$ and $(\delta_{s, t}.e) \otimes \mathbf{1}$.
        
          \begin{flalign*}
            & \mathbf{1} \otimes (u + v) \reduce \left \{
              \begin{aligned}
                & \textsc{(TsrIdL)} && \\
                & \textsc{(TsrDistR)} && \mathbf{1} \otimes u + \mathbf{1} \otimes v
              \end{aligned}
              \right \} \reduce u + v & 
          \end{flalign*}
          \textbf{Remark:} Similar for $(u + v) \otimes \mathbf{1}$.

          \begin{flalign*}
            & \mathbf{1} \otimes e \reduce \left \{
              \begin{aligned}
                & \textsc{(TsrIdL)} && \\
                & \textsc{(TsrComm)} && e \otimes \mathbf{1}
              \end{aligned}
              \right \} \reduce e & 
          \end{flalign*}
          \textbf{Remark:} Similar for $e \otimes \mathbf{1}$. Also similar for critical pairs between \textsc{(TsrIdL)}, \textsc{(TsrIdR)} and \textsc{(TsrAssoc)}, \textsc{(TsrSort)}.


        %%%%%%%%%%%%%%%%%%%%%%%%%%%%%%%%%%%%%%%%%%%%%%%%%%%%%%%%%%%%%%%%
        \item \textbf{critical pairs containing} \textsc{(TsrScrL)}/\textsc{(TsrScrR)}/\textsc{(TsrDeltaL)}/\textsc{(TsrDeltaR)}

        \begin{flalign*}
          & (\alpha.u) \otimes v \left \{
            \begin{aligned}
              & \textsc{(TsrScrL)} && \alpha.(u \otimes v) \\
              & \textsc{(TsrComm)} && v \otimes (\alpha.u) \reduce \alpha.(v \otimes u)
            \end{aligned}
            \right \} \reduce 
            \left \{
            \begin{aligned}
              & \alpha.(u \otimes v) & (v <_{RPO} u) \\
              & \alpha.(v \otimes u) & (u <_{RPO} v) 
            \end{aligned}
            \right . 
            \qquad (v <_{RPO} \alpha.u) &
        \end{flalign*}
        \textbf{Remark:} The two expressions will join at either $\alpha.(u \otimes v)$ or $\alpha.(v \otimes u)$, depending on the order of $u$ and $v$. Similar for other critical pairs between \textsc{(TsrScrL)}, \textsc{(TsrScrR)} and \text{(TsrComm)}, \textsc{(TsrAssoc)}, \textsc{(TsrSort)}.


        %%%%%%%%%%%%%%%%%%%%%%%%%%%%%%%%%%%%%%%%%%%%%%%%%%%%%%%%%%%%%%%%
        \item \textbf{critical pairs containing} \textsc{(TsrDistL)}/\textsc{(TsrDistR)}
        
        \begin{flalign*}
          & (a + b)\otimes (u + v) \reduce \left \{
            \begin{aligned}
              & \textsc{(TsrDistL)} && (a + b) \otimes u + (a + b) \otimes v \\
              & \textsc{(TsrDistR)} && a \otimes (u + v) + b \otimes (u + v)
            \end{aligned}
            \right \} \reduce a \otimes u + a \otimes v + b \otimes u + b \otimes v & 
        \end{flalign*}


        %%%%%%%%%%%%%%%%%%%%%%%%%%%%%%%%%%%%%%%%%%%%%%%%%%%%%%%%%%%%%%%%
        \item \textbf{critical pairs between} \textsc{(TsrComm)-(TsrAssoc)}
        
        The critical pairs between \textsc{(TsrComm)} and \textsc{(TsrAssoc)} diverge from the expression $A \otimes (B \otimes C)$, with corresponding side conditions on their types being satisfied. There are three different cases (merging the symmetric ones):

        \begin{center}
          \includegraphics*[width=0.8\textwidth]{fig/CommAssocPairs.png}
        \end{center}

        Take the first case as an example. The assumptions are $\Gamma \vdash B : [\unit, \unit]$ and $B <_{RPO} C$. The confluence proof is shown in the following graph. Similar for other cases.
        \begin{center}
          \includegraphics*[width=0.8\textwidth]{fig/CommAssocCase1.png}
        \end{center}


        %%%%%%%%%%%%%%%%%%%%%%%%%%%%%%%%%%%%%%%%%%%%%%%%%%%%%%%%%%%%%%%%
        \item \textbf{critical pairs between} \textsc{(TsrComm)-(TsrSort1)}
        
        The critical pairs have three different cases (merging the symmetric ones):
        
        \begin{center}
          \includegraphics*[width=0.8\textwidth]{fig/CommSort1Pairs.png}
        \end{center}

        Take the first case as an example. The assumptions are $\Gamma \vdash A : [\tau, \unit]$, $\Gamma \vdash B : [\unit, \sigma]$, $\Gamma \vdash C : [\rho, \unit]$, $A <_{RPO} B < _{RPO} < C$. The confluence proof is shown in the following graph. Similar for other cases.
        \begin{center}
          \includegraphics*[width=0.8\textwidth]{fig/CommSort1Case1.png}
        \end{center}


        %%%%%%%%%%%%%%%%%%%%%%%%%%%%%%%%%%%%%%%%%%%%%%%%%%%%%%%%%%%%%%%%
        \item \textbf{critical pairs between} \textsc{(TsrComm)-(TsrSort2)}
        
        Similar to \textsc{(TsrComm)-(TsrSort1)} critical pairs.

        %%%%%%%%%%%%%%%%%%%%%%%%%%%%%%%%%%%%%%%%%%%%%%%%%%%%%%%%%%%%%%%%
        \item \textbf{critical pairs of} \textsc{(TsrAssoc)}

        \begin{flalign*}
          & A \otimes (B \otimes (C \otimes D)) \reduce \left \{
            \begin{aligned}
              & \textsc{(TsrAssoc)} && A \otimes (B \otimes C \otimes D) \reduce \cdots \\
              & \textsc{(TsrAssoc)} && A \otimes B \otimes (C \otimes D) \reduce \cdots
            \end{aligned}
            \right \} \reduce A \otimes B \otimes C \otimes D & 
        \end{flalign*}

        %%%%%%%%%%%%%%%%%%%%%%%%%%%%%%%%%%%%%%%%%%%%%%%%%%%%%%%%%%%%%%%%
        \item \textbf{critical pairs between} \textsc{(TsrAssoc)-(TsrSort1)}
        
        \begin{itemize}
          \item $(A \otimes B) \otimes (C \otimes D)$
          \item $(A \otimes (B \otimes C)) \otimes D$
        \end{itemize}

        %%%%%%%%%%%%%%%%%%%%%%%%%%%%%%%%%%%%%%%%%%%%%%%%%%%%%%%%%%%%%%%%
        \item \textbf{critical pairs of} \textsc{(TsrSort1)}

        The critical pairs comes from $A \otimes B \otimes C \otimes D$. There are three cases:

        \begin{center}
          \includegraphics*[width=0.8\textwidth]{fig/Sort1Sort1Pairs.png}
        \end{center}

        Take the first case as an example. The assumptions are $\Gamma \vdash A : [\tau_A, \sigma_A]$, $\Gamma \vdash B : [\tau_B, \sigma_B]$, $\Gamma \vdash C : [\unit, \unit]$, $\Gamma \vdash D : [\tau_D, \sigma_D]$, $B <_{RPO} C < _{RPO} < D$. The confluence proof is shown in the following graph. Similar for other cases.

        \begin{center}
          \includegraphics*[width=0.8\textwidth]{fig/Sort1Sort1Case1.png}
        \end{center}

        %%%%%%%%%%%%%%%%%%%%%%%%%%%%%%%%%%%%%%%%%%%%%%%%%%%%%%%%%%%%%%%%
        \item \textbf{critical pairs between} \textsc{(TsrSort1)-(TsrSort2)}

        \begin{itemize}
          \item $A \otimes B \otimes C$
          \item $A \otimes B \otimes C \otimes D$
        \end{itemize}

        %%%%%%%%%%%%%%%%%%%%%%%%%%%%%%%%%%%%%%%%%%%%%%%%%%%%%%%%%%%%%%%%
        \item \textbf{critical pairs of} \textsc{(TsrSort2)}

        The critical pairs comes from $A \otimes B \otimes C \otimes D$. There are five cases:

        \begin{center}
          \includegraphics*[width=0.8\textwidth]{fig/Sort2Sort2Pairs.png}
        \end{center}

        Take the first case as an example. The assumptions are $\Gamma \vdash A : [\tau_A, \sigma_A]$, $\Gamma \vdash B : [\tau_B, \sigma_B]$, $\Gamma \vdash C : [\unit, \unit]$, $\Gamma \vdash D : [\tau_D, \sigma_D]$, $B <_{RPO} C < _{RPO} < D$. The confluence proof is shown in the following graph. Similar for other cases.

        \begin{center}
          \includegraphics*[width=0.8\textwidth]{fig/Sort1Sort1Case1.png}
        \end{center}
      \end{itemize}

\end{proof}