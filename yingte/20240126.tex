\chapter{20240112}

\newcommand*{\unit}{\texttt{unit}}
\newcommand*{\utt}{\texttt{tt}}
\newcommand*{\fst}{\texttt{fst}}
\newcommand*{\snd}{\texttt{snd}}
\newcommand*{\reduce}{\ \triangleright\ }
\newcommand*{\reducefrom}{\ \triangleleft\ }

\newcommand*{\zeroK}[1]{\mathbf{0}_{\mathcal{K}(#1)}}
\newcommand*{\zeroB}[1]{\mathbf{0}_{\mathcal{B}(#1)}}
\newcommand*{\zeroO}[1]{\mathbf{0}_{\mathcal{O}(#1)}}

\section{Dirac Notation}

We want to study the theory of Dirac notations modulo the scalars. This idea is inspired by the Lineal paper \cite{Arrighi2017}.

\begin{definition}[complex scalar rewrite system]
  A \textbf{complex scalar rewriting system} is a rewrite system $S$ on a language containing at least the symbols $0, 1, +, \times, *$, such that:
  \begin{itemize}
    \item $S$ is terminating and ground confluent,
    \item $+$ and $\times$ are AC-symbols,
    \item for all closed terms $\alpha$, $\beta$ and $\gamma$, the pair of terms
      \begin{itemize}
        \item $0 + \alpha$ and $\alpha$
        \item $0 \times \alpha$ and $0$,
        \item $1 \times \alpha$ and $\alpha$,
        \item $\alpha \times (\beta + \gamma)$ and $\alpha \times \beta + \alpha \times \gamma$,
        \item $(\alpha + \beta)^*$ and $\alpha^* + \beta^*$,
        \item $(\alpha \times \beta)^*$ and $\alpha^* \times \beta^*$,
        \item $(\alpha^*)^*$ and $\alpha$,
      \end{itemize}
      have the same normal forms,
    \item $0$ and $1$ are normal terms.
  \end{itemize}
\end{definition}

We can give a smallest instantiation $S_0$ of a complext scalar rewriting system. This will help in the proof of confluence of the Dirac notation rewriting system by the avatar lamma.

\begin{definition}[The rewrite system $S_0$]
  The system $S_0$ is defined by the following rules:
  \begin{gather*}
    0 + \alpha \reduce \alpha\\
    0 \times \alpha \reduce 0\\
    1 \times \alpha \reduce \alpha\\
    \alpha \times (\beta + \gamma) \reduce \alpha \times \beta + \alpha \times \gamma\\
    (\alpha + \beta)^* \reduce \alpha^* + \beta^*\\
    (\alpha \times \beta)^* \reduce \alpha^* \times \beta^*\\
    (\alpha^*)^* \reduce \alpha
  \end{gather*}
  where $+$ and $\times$ are AC-symbols.
\end{definition}

The Dirac notation theory is also studied modulo the types and terms for the atomic basis of Hilbert spaces, which are defined by basis rewrite systems.

\begin{definition}[atomic basis rewrite system]
  An \textbf{atomic basis rewrite system} $B$ is a terminating and ground confluent rewrite system.
\end{definition}

\subsection{Syntax and Typing}

Assume we are given a complex scalar rewrite system $S$ and an atomic basis rewrite system $B$ in the following.

\begin{definition}[language of scalars]
  The \textbf{language of scalars} is a first order language containing all the symbols in $S$.
\end{definition}


\begin{definition}[language of atomic basis]
  The \textbf{language of atomic basis} is a first order language containing all the symbols in $B$.
\end{definition}


\begin{definition}[term and type of basis]
  The \textbf{term of basis} with pairing is defined:
  \begin{align*}
    t ::= x\ |\ b\ |\ (t, t)\ |\ \fst\ t\ |\ \snd\ t
  \end{align*}
  Here $b$ is an atomic basis and $x$ is a variable. The \textbf{type of basis} is defined by
  \begin{align*}
    \tau ::= T\ |\ (\tau * \tau)
  \end{align*}
  Here $T$ is the type for atomic bases.
\end{definition}

\begin{definition}[typing rules]
  A typing assumption has the form $x : \tau$, meaning variable $x$ has the type $\tau$. A typing context $\Gamma$ consists of typing assumptions and each variable appears only once.

  A typing judgement $\Gamma \vdash e : \sigma$ indicates that $e$ is a term of type $\sigma$ in context $\Gamma$. The well-typed terms are defined by the following rules:
  \begin{gather*}
    \frac{x : \sigma \in \Gamma}{\Gamma \vdash x : \sigma}
    \qquad 
    \Gamma \vdash b : T\\
    \\
    \frac{\Gamma \vdash t_1 : \tau \qquad \Gamma \vdash t_2 : \sigma}{\Gamma \vdash (t_1, t_2) : ( \tau * \sigma )} \\
    \\
    \frac{\Gamma \vdash t : ( \tau * \sigma )}{\Gamma \vdash \fst\ t : \tau}
    \qquad
    \frac{\Gamma \vdash t : ( \tau * \sigma )}{\Gamma \vdash \snd\ t : \sigma}
  \end{gather*}


\begin{definition}[language of Scalar]
  \begin{align*}
    S ::= x\ |\ \alpha\ |\ \delta_{t, t}\ |\ S + S\ |\ S \times S\ |\ S^*\ |\ B \cdot K
  \end{align*}
  Here $+$ and $\times$ are AC-symbols, and $\delta_{s, t} = \delta_{t, s}$.
\end{definition}

\begin{definition}[typing rules for Scalar]
  \begin{gather*}
    \frac{x : \mathcal{S} \in \Gamma}{\Gamma \vdash x : \mathcal{S}}
    \qquad 
    \Gamma \vdash \alpha : \mathcal{S}
    \qquad
    \frac{\Gamma \vdash s : \tau \qquad \Gamma \vdash t : \tau}{\Gamma \vdash \delta_{s, t} : \mathcal{S}} \\
    \\
    \frac{\Gamma \vdash S_1 : \mathcal{S} \qquad \Gamma \vdash S_2 : \mathcal{S}}{\Gamma \vdash S_1 + S_2 : \mathcal{S}}
    \qquad
    \frac{\Gamma \vdash S_1 : \mathcal{S} \qquad \Gamma \vdash S_2 : \mathcal{S}}{\Gamma \vdash S_1 \times S_2 : \mathcal{S}} \\
    \\
    \frac{\Gamma \vdash S : \mathcal{S}}{\Gamma \vdash S^* : \mathcal{S}} \\
    \\
    \frac{\Gamma \vdash B : \mathcal{B}(\tau) \qquad \Gamma \vdash K : \mathcal{K}(\tau)}{\Gamma \vdash B \cdot K : \mathcal{S}}
  \end{gather*}
\end{definition}


\begin{definition}[language of Ket]
  \begin{align*}
    K ::= x\ |\ \zeroK{\tau}\ |\ \ket{t}\ |\ K^*\ |\ B^\top\ |\ S.K\ |\ K + K\ |\ O \cdot K\ |\ K \otimes K
  \end{align*}
  Here $+$ is AC-symbol.
\end{definition}

\yx{Symmetricity!}

\begin{definition}[typing rules for Ket]
  \begin{gather*}
    \frac{x : \mathcal{K}(\tau) \in \Gamma}{\Gamma \vdash x : \mathcal{K}(\tau)}
    \qquad
    \Gamma \vdash \zeroK{\tau} : \mathcal{K}(\tau)
    \qquad
    \frac{\Gamma \vdash t : \tau}{\Gamma \vdash \ket{t} : \mathcal{K}(\tau)} \\
    \\
    \frac{\Gamma \vdash K : \mathcal{K}(\tau)}{\Gamma \vdash K^* : \mathcal{K}(\tau)}
    \qquad
    \frac{\Gamma \vdash B : \mathcal{B}(\tau)}{\Gamma \vdash B^\top : \mathcal{K}(\tau)}
    \qquad
    \frac{\Gamma \vdash S : \mathcal{S} \qquad \Gamma \vdash K : \mathcal{K}(\tau)}{\Gamma \vdash S.K : \mathcal{K}(\tau)} \\
    \\
    \frac{\Gamma \vdash K_1 : \mathcal{K}(\tau) \qquad \Gamma \vdash K_2 : \mathcal{K}(\tau)}{\Gamma \vdash K_1 + K_2 : \mathcal{K}(\tau)} \\
    \\
    \frac{\Gamma \vdash O : \mathcal{O}(\tau, \sigma) \qquad \Gamma \vdash K : \mathcal{K}(\sigma)}{\Gamma \vdash O \cdot K : \mathcal{K}(\tau)} \\
    \\
    \frac{\Gamma \vdash K_1 : \mathcal{K}(\tau_1) \qquad \Gamma \vdash K_2 : \mathcal{K}(\tau_2)}{\Gamma \vdash K_1 \otimes K_2 : \mathcal{K}(\tau_1 * \tau_2)}
  \end{gather*}
\end{definition}


\begin{definition}[language of Bra]
  \begin{align*}
    B ::= x\ |\ \zeroB{\tau}\ |\ \bra{t}\ |\ B^*\ |\ K^\top\ |\ S.B\ |\ B + B\ |\ B \cdot O\ |\ B \otimes B
  \end{align*}
  Here $+$ is AC-symbol.
\end{definition}


\begin{definition}[typing rules for Bra]
  \begin{gather*}
    \frac{x : \mathcal{B}(\tau) \in \Gamma}{\Gamma \vdash x : \mathcal{B}(\tau)}
    \qquad
    \Gamma \vdash \zeroB{\tau} : \mathcal{B}(\tau)
    \qquad
    \frac{\Gamma \vdash t : \tau}{\Gamma \vdash \bra{t} : \mathcal{B}(\tau)} \\
    \\
    \frac{\Gamma \vdash B : \mathcal{B}(\tau)}{\Gamma \vdash B^* : \mathcal{B}(\tau)}
    \qquad
    \frac{\Gamma \vdash K : \mathcal{K}(\tau)}{\Gamma \vdash K^\top : \mathcal{B}(\tau)}
    \qquad
    \frac{\Gamma \vdash S : \mathcal{S} \qquad \Gamma \vdash B : \mathcal{B}(\tau)}{\Gamma \vdash S.B : \mathcal{B}(\tau)} \\
    \\
    \frac{\Gamma \vdash B_1 : \mathcal{B}(\tau) \qquad \Gamma \vdash B_2 : \mathcal{B}(\tau)}{\Gamma \vdash B_1 + B_2 : \mathcal{B}(\tau)} \\
    \\
    \frac{\Gamma \vdash B : \mathcal{B}(\tau) \qquad \Gamma \vdash O : \mathcal{O}(\tau, \sigma)}{\Gamma \vdash B \cdot O : \mathcal{B}(\sigma)} \\
    \\
    \frac{\Gamma \vdash B_1 : \mathcal{B}(\tau_1) \qquad \Gamma \vdash B_2 : \mathcal{B}(\tau_2)}{\Gamma \vdash B_1 \otimes B_2 : \mathcal{B}(\tau_1 * \tau_2)}
  \end{gather*}
\end{definition}


\begin{definition}[language of Operators]
  \begin{align*}
    O ::= x\ |\ \zeroO{\tau, \sigma}\ |\ K \otimes B\ |\ O^*\ |\ O^\top\ |\ S.O\ |\ O + O\ |\ O \cdot O\ |\ O \otimes O
  \end{align*}
  Here $+$ is AC-symbol.
\end{definition}

\begin{gather*}
  \frac{x : \mathcal{O}(\tau, \sigma) \in \Gamma}{\Gamma \vdash x : \mathcal{O}(\tau, \sigma)}
  \qquad
  \Gamma \vdash \zeroO{\tau, \sigma} : \mathcal{O}(\tau, \sigma) 
  \qquad
  \frac{\Gamma \vdash K : \mathcal{K}(\tau) \qquad \Gamma \vdash B : \mathcal{B}(\sigma)}{\Gamma \vdash K \otimes B : \mathcal{O}(\tau, \sigma)} \\
  \\
  \frac{\Gamma \vdash O : \mathcal{O}(\tau, \sigma)}{\Gamma \vdash O^* : \mathcal{O}(\tau, \sigma)}
  \qquad
  \frac{\Gamma \vdash O : \mathcal{O}(\tau, \sigma)}{\Gamma \vdash O^\top : \mathcal{O}(\sigma, \tau)}
  \qquad
  \frac{\Gamma \vdash S : \mathcal{S} \qquad \Gamma \vdash O : \mathcal{O}(\tau, \sigma)}{\Gamma \vdash S.O : \mathcal{O}(\tau, \sigma)} \\
  \\
  \frac{\Gamma \vdash O_1 : \mathcal{O}(\tau, \sigma) \qquad \Gamma \vdash O_2 : \mathcal{O}(\tau, \sigma)}{\Gamma \vdash O_1 + O_2 : \mathcal{O}(\tau, \sigma)} \\
  \\
  \frac{\Gamma \vdash O_1 : \mathcal{O}(\tau, \sigma) \qquad \Gamma \vdash O_2 : \mathcal{O}(\sigma, \rho)}{\Gamma \vdash O_1 \cdot O_2 : \mathcal{O}(\tau, \rho)} \\
  \\
  \frac{\Gamma \vdash O_1 : \mathcal{O}(\tau_1, \sigma_1) \qquad \Gamma \vdash O_2 : \mathcal{O}(\tau_2, \sigma_2)}{\Gamma \vdash O_1 \cdot O_2 : \mathcal{O}(\tau_1 * \tau_2, \rho_1 * \rho_2)}
\end{gather*}



\yx{Types determines how to overload the operators. The essence of Dirac notation is to decide the type of the term and then overload the operators.}

\end{definition}

% \begin{claim}
%   For any term $e$ of Dirac lambda calculus in any context $\Gamma$, there exists at most one type $\tau$ that satisfies $\Gamma \vdash e : \tau$. The types of all terms are computable (if exist).
% \end{claim}

\subsection{Equational Theory for Delta Operator}

$$
  \frac{(\bigwedge_i s_i =_\textsf{BASIC} t_i) \leftrightarrow (\bigwedge_i s_i' =_\textsf{BASIC} t_i') \textrm{ is valid}}{\prod_i \delta_{s_i, t_i} = \prod_i \delta_{s_i', t_i'}}
$$

\subsection{Reduction Rules}

\subsubsection*{\textsf{BASIS}}
\begin{align*}
    & \textsc{(Proj1)} && \Gamma \vdash \fst\ (e_1, e_2) \reduce e_1
    && \textsc{(Proj2)} && \Gamma \vdash \snd\ (e_1, e_2) \reduce e_2  \\
    \\
    & \textsc{(Pair)} && \Gamma \vdash (\fst\ e, \snd\ e)\reduce e
\end{align*}


\subsubsection*{\textsf{SCAL-DELTA}}
\begin{align*}
  & \textsc{(Delta1)} && 
  \frac{\bigwedge_i s_i =_\textsf{BASIC} t_i \textrm{ is valid}}{\Gamma \vdash \prod_i \delta_{s_i, t_i} \reduce 1}
  \\
  \\
  & \textsc{(Delta0)} && 
  \frac{\bigwedge_i s_i =_\textsf{BASIS} t_i \textrm{ is not satisfiable}}{\Gamma \vdash \prod_i \delta_{s_i, t_i} \reduce 0}
\end{align*}


\subsubsection*{\textsf{SCAL}}
\begin{align*}
  & \Gamma \vdash \alpha + \beta \reduce \alpha + \beta\ \textrm{(addition in $\mathfrak{S}$)} \\
  & \Gamma \vdash S + S \reduce (1 + 1) \times S \ \textrm{(addition in $\mathfrak{S}$)} \\
  & \Gamma \vdash \alpha \times S + S \reduce (\alpha + 1) \times S\ \textrm{(addition in $\mathfrak{S}$)} \\
  & \Gamma \vdash \alpha \times S + \beta \times S \reduce (\alpha + \beta) \times S \ \textrm{(addition in $\mathfrak{S}$)} \\
  \\
  & \Gamma \vdash \alpha \times \beta \reduce \alpha \times \beta\ \textrm{(multiplication in $\mathfrak{S}$)} \\
  & \Gamma \vdash S_1 \times (S_2 + S_3) \reduce S_1 \times S_2 + S_1 \times S_3\\
  \\
  & \Gamma \vdash \alpha^* \reduce \alpha^*\ \textrm{(conjugate in $\mathfrak{S}$)} \\
  & \Gamma \vdash \delta_{s, t}^* \reduce \delta_{s, t} \\
  & \Gamma \vdash (S_1 + S_2)^* \reduce S_1^* + S_2^* \\
  & \Gamma \vdash (S_1 \times S_2)^* \reduce S_1^* \times S_2^* \\
  & \Gamma \vdash (S^*)^* \reduce S \\
  & \Gamma \vdash (B \cdot K)^* \reduce B^* \cdot K^* \\
  \\
  & \Gamma \vdash \mathbf{0}_\mathcal{K} \cdot B \reduce 0 \\
  & \Gamma \vdash K \cdot \mathbf{0}_\mathcal{B} \reduce 0 \\
  & \Gamma \vdash (S.B) \cdot K \reduce S.(B \cdot K) \\
  & \Gamma \vdash B \cdot (S.K) \reduce S.(B \cdot K) \\
  & \Gamma \vdash (B_1 + B_2) \cdot K \reduce B_1 \cdot K + B_2 \cdot K \\
  & \Gamma \vdash B \cdot (K_1 + K_2) \reduce B \cdot K_1 + B \cdot K_2 \\
  & \Gamma \vdash \bra{s} \cdot \ket{t} \reduce \delta_{s, t} \\
  & \Gamma \vdash (B_1 \otimes B_2) \cdot \ket{t} \reduce (B_1 \cdot \ket{\fst\ t}) \times (B_2 \cdot \ket{\snd\ t}) \\
  & \Gamma \vdash \bra{t} \cdot (K_1 \otimes K_2) \reduce (\bra{\fst\ t} \cdot K_1) \times (\bra{\snd\ t} \cdot K_2) \\
  & \Gamma \vdash (B_1 \otimes B_2) \cdot (K_1 \otimes K_2) \reduce (B_1 \cdot K_1) \times (B_2 \cdot K_2)
\end{align*}

\textbf{Remark:} Notice that we don't consider $\Gamma \vdash (S.B) \cdot K \reduce (S^*).(B \cdot K)$. Inner product is linear (not conjugate linear) on $B$, because $B$ is already in the dual space.

\yx{$\delta_{i,j}\times \delta_{j, k} = \delta_{i, j} \times \delta_{i, k}$? This is really troublesome.}

\yx{I believe the best way is consider it as a proposition. But how to deal with $\sum_i \delta_{i, j}$ then? -- It should be summing on the set that satisfies the proposition.}

\yx{At least we should draw a line between what can be decided and what can't, on this delta operator.}


\subsection{Reduction Rules for Ket}

\subsubsection*{\textsf{KET-CONJ}}
\begin{align*}
  & \textsc{(Zero*)} && \Gamma \vdash \zeroK{\tau}^* \reduce \zeroK{\tau} \\
  & \textsc{(Ket*)} && \Gamma \vdash \ket{t}^* \reduce \ket{t} \\
  & \textsc{(Double*)} && \Gamma \vdash (K^*)^* \reduce K \\
  & \textsc{(T*)} && \Gamma \vdash (B^\top)^* \reduce(B^*)^\top \\
  & \textsc{(Scr*)} && \Gamma \vdash (S.K)^* \reduce (S^*).(K^*) \\
  & \textsc{(Add*)} && \Gamma \vdash (K_1 + K_2)^* \reduce K_1^* + K_2^* \\
  & \textsc{(Mul*)} && \Gamma \vdash (O \cdot K)^* \reduce O^* \cdot K^* \\
  & \textsc{(Tsr*)} && \Gamma \vdash (K_1 \otimes K_2)^* \reduce K_1^* \otimes K_2^*
\end{align*}

\subsubsection*{\textsf{KET-TRANS}}
\begin{align*}
  & \textsc{(ZeroT)} && \Gamma \vdash \zeroB{\tau}^\top \reduce \zeroK{\tau} \\
  & \textsc{(BraT)} && \Gamma \vdash \bra{t}^\top \reduce \ket{t}\\
  & \textsc{(DoubleT)} && \Gamma \vdash (K^\top)^\top \reduce K \\
  & \textsc{(ScrT)} && \Gamma \vdash (S.B)^\top \reduce S.(B^\top) \\
  & \textsc{(AddT)} && \Gamma \vdash (B_1 + B_2)^\top \reduce B_1^\top + B_2^\top\\
  & \textsc{(MulT)} && \Gamma \vdash (B \cdot O)^\top \reduce O^\top \cdot B^\top \\
  & \textsc{(TsrT)} && \Gamma \vdash (B_1 \otimes B_2)^\top \reduce B_1^\top \otimes B_2^\top
\end{align*}

\subsubsection*{\textsf{KET-SCAL}}
\begin{align*}
  & \textsc{(Scr0)} && \Gamma \vdash 0.K \reduce \textbf{0}_{\mathcal{K}}
  \qquad
  \textcolor{red}{\Gamma \vdash 0.K \reduce \zeroK{\tau}\ \textrm{(where $\Gamma \vdash K : \mathcal{K}(\tau)$)}} \\
  & \textsc{(Scr1)} && \Gamma \vdash 1.K \reduce K \\
  & \textsc{(ScrZero)} && \Gamma \vdash S.\zeroK{\tau} \reduce \zeroK{\tau} \\
  & \textsc{(ScrScr)} && \Gamma \vdash S_1.(S_2.K) \reduce (S_1 \times S_2).K \\
  & \textsc{(ScrDist)} && \Gamma \vdash S.(K_1 + K_2) \reduce S.K_1 + S.K_2
\end{align*}

\subsubsection*{\textsf{KET-ADD}}
\begin{align*}
  & \textsc{(AddZero)} && \Gamma \vdash K + \zeroK{\tau} \reduce K \\
  & \textsc{(Fac0)} && \Gamma \vdash K + K \reduce (1 + 1).K \\
  & \textsc{(Fac1)} && \Gamma \vdash S.K + K \reduce (S + 1).K \\
  & \textsc{(Fac2)} && \Gamma \vdash S_1.K + S_2.K \reduce (S_1 + S_2).K
\end{align*}

\subsubsection*{\textsf{KET-MUL}}
\begin{align*}
  & \textsc{(MulZeroL)} && \Gamma \vdash \zeroO{\tau, \sigma} \cdot K \reduce \zeroK{\tau} \\
  & \textsc{(MulZeroR)} && 
  \Gamma \vdash O \cdot \mathbf{0}_{\mathcal{K}} \reduce \mathbf{0}_{\mathcal{K}}
  \qquad
  \textcolor{red}{{\Gamma \vdash O \cdot \zeroK{\sigma} \reduce \zeroK{\tau}}\ \textrm{(where $\Gamma \vdash O : \mathcal{O}(\tau, \sigma)$)}} \\
  & \textsc{(MulScrL)} && \Gamma \vdash (S.O) \cdot K \reduce S.(O \cdot K) \\
  & \textsc{(MulScrR)} && \Gamma \vdash O \cdot (S.K) \reduce S.(O \cdot K) \\
  & \textsc{(MulDistL)} && \Gamma \vdash (O_1 + O_2) \cdot K \reduce O_1 \cdot K + O_2 \cdot K \\
  & \textsc{(MulDistR)} && \Gamma \vdash O \cdot (K_1 + K_2) \reduce O \cdot K_1 + O \cdot K_2 \\
  & \textsc{(MulReFac)} && \Gamma \vdash (K_1 \otimes B) \cdot K_2 \reduce (B \cdot K_2).K_1 \\
  & \textsc{(MulAssoc)} && \Gamma \vdash (O_1 \cdot O_2) \cdot K \reduce O_1 \cdot (O_2 \cdot K) \\
  & \textsc{(MulAssoc2)} && \Gamma \vdash (O_1 \otimes O_2) \cdot ((O_1' \otimes O_2') \cdot K) \reduce ((O_1 \cdot O_1') \otimes (O_2 \cdot O_2')) \cdot K \\
  & \textsc{(MulReFac)} && \Gamma \vdash (O_1 \otimes O_2) \cdot \ket{t} \reduce (O_1 \cdot \ket{\fst\ t}) \otimes (O_2 \cdot \ket{\snd\ t}) \\
  & \textsc{(MulReFac)} && \Gamma \vdash (O_1 \otimes O_2) \cdot (K_1 \otimes K_2) \reduce (O_1 \cdot K_1) \otimes (O_2 \cdot K_1) \\
\end{align*}

\textbf{Remark:} The \textsc{(MulAssoc2)} rule is for completion.

\subsubsection*{\textsf{KET-TSR}}
\begin{align*}
  & \textsc{(TsrZero)} && \Gamma \vdash \mathbf{0}_\mathcal{K} \otimes K \reduce \mathbf{0}_\mathcal{K}
  \qquad
  \textcolor{red}{\Gamma \vdash \zeroK{\tau} \otimes K \reduce \zeroK{\tau * \sigma}\ \textrm{(where $\Gamma \vdash K : \mathcal{K}(\sigma)$)}}\\
  & \textsc{(TsrZero)} && \Gamma \vdash K \otimes \mathbf{0}_\mathcal{K} \reduce \mathbf{0}_\mathcal{K}
  \qquad
  \textcolor{red}{\Gamma \vdash K \otimes \zeroK{\sigma} \reduce \zeroK{\tau * \sigma}\ \textrm{(where $\Gamma \vdash K : \mathcal{K}(\tau)$)}} \\
  & \textsc{(KetPair)} && \Gamma \vdash \ket{s} \otimes \ket{t}\reduce\ket{(s, t)} \\
  & \textsc{(TsrScrL)} && \Gamma \vdash (S.K_1) \otimes K_2 \reduce S.(K_1 \otimes K_2) \\
  & \textsc{(TsrScrR)} && \Gamma \vdash K_1 \otimes (S.K_2) \reduce S.(K_1 \otimes K_2) \\
  & \textsc{(TsrDistL)} && \Gamma \vdash (K_1 + K_2) \otimes K_3 \reduce K_1 \otimes K_3 + K_2 \otimes K_3 \\
  & \textsc{(TsrDistR)} && \Gamma \vdash K_1 \otimes (K_2 + K_3) \reduce K_1 \otimes K_2 + K_1 \otimes K_3 \\
\end{align*}

\subsection{Reduction Rules for Operator}

\subsubsection*{\textsf{OPT-OUTER}}
\begin{align*}
  & \textsc{(TsrZero)} && \Gamma \vdash \mathbf{0}_\mathcal{K} \otimes B \reduce \mathbf{0}_\mathcal{O}
  \qquad
  \textcolor{red}{\Gamma \vdash \zeroK{\tau} \otimes B \reduce \zeroO{\tau, \sigma}\ \textrm{(where $\Gamma \vdash B : \mathcal{B}(\sigma)$)}}\\
  & \textsc{(TsrZero)} && \Gamma \vdash K \otimes \mathbf{0}_\mathcal{B} \reduce \mathbf{0}_\mathcal{O} 
  \qquad
  \textcolor{red}{\Gamma \vdash K \otimes \zeroB{\sigma} \reduce \zeroO{\tau, \sigma}\ \textrm{(where $\Gamma \vdash K : \mathcal{K}(\tau)$)}}\\
  & \textsc{(TsrScrL)} && \Gamma \vdash (S.K) \otimes B \reduce S.(K \otimes B) \\
  & \textsc{(TsrScrR)} && \Gamma \vdash K \otimes (S.B) \reduce S.(K \otimes B) \\
  & \textsc{(TsrDistL)} && \Gamma \vdash (K_1 + K_2) \otimes B \reduce K_1 \otimes B + K_2 \otimes B \\
  & \textsc{(TsrDistR)} && \Gamma \vdash K \otimes (B_1 + B_2) \reduce K \otimes B_1 + K \otimes B_2 \\
\end{align*}


\subsubsection*{\textsf{OPT-CONJ}}
\begin{align*}
  & \textsc{(Zero*)} && \Gamma \vdash \zeroO{\tau, \sigma}^* \reduce \zeroO{\tau, \sigma} \\
  & \textsc{(Outer*)} && \Gamma \vdash (K \otimes B)^* \reduce K^* \otimes B^* \\
  & \textsc{(Double*)} && \Gamma \vdash (O^*)^* \reduce O \\
  & \textsc{(O*)} && \Gamma \vdash (O^\top)^* \reduce(O^*)^\top \\
  & \textsc{(Scr*)} && \Gamma \vdash (S.O)^* \reduce (S^*).(O^*) \\
  & \textsc{(Add*)} && \Gamma \vdash (O_1 + O_2)^* \reduce O_1^* + O_2^* \\
  & \textsc{(Mul*)} && \Gamma \vdash (O_1 \cdot O_2)^* \reduce O_1^* \cdot O_2^* \\
  & \textsc{(Tsr*)} && \Gamma \vdash (O_1 \otimes O_2)^* \reduce O_1^* \otimes O_2^*
\end{align*}

\subsubsection*{\textsf{OPT-TRANS}}
\begin{align*}
  & \textsc{(ZeroT)} && \Gamma \vdash \zeroO{\tau, \sigma}^\top \reduce \zeroO{\sigma, \tau} \\
  & \textsc{(OuterT)} && \Gamma \vdash (K \otimes B)^\top \reduce B^\top \otimes K^\top \\
  & \textsc{(DoubleT)} && \Gamma \vdash (O^\top)^\top \reduce O \\
  & \textsc{(ScrT)} && \Gamma \vdash (S.O)^\top \reduce S.(O^\top) \\
  & \textsc{(AddT)} && \Gamma \vdash (O_1 + O_2)^\top \reduce O_1^\top + O_2^\top\\
  & \textsc{(MulT)} && \Gamma \vdash (O_1 \cdot O_2)^\top \reduce O_2^\top \cdot O_1^\top \\
  & \textsc{(TsrT)} && \Gamma \vdash (O_1 \otimes O_2)^\top \reduce O_1^\top \otimes O_2^\top
\end{align*}

\subsubsection*{\textsf{OPT-SCAL}}
\begin{align*}
  & \textsc{(Scr0)} && \Gamma \vdash 0.O \reduce \mathbf{0}_\mathcal{O}
  \qquad
  \textcolor{red}{\Gamma \vdash 0.O \reduce \zeroO{\tau, \sigma}\ \textrm{(where $\Gamma \vdash O : \mathcal{O}(\tau, \sigma)$)}} \\
  & \textsc{(Scr1)} && \Gamma \vdash 1.O \reduce O \\
  & \textsc{(ScrZero)} && \Gamma \vdash S.\zeroO{\tau, \sigma} \reduce \zeroO{\tau, \sigma} \\
  & \textsc{(ScrScr)} && \Gamma \vdash S_1.(S_2.O) \reduce (S_1 \times S_2).O \\
  & \textsc{(ScrDist)} && \Gamma \vdash S.(O_1 + O_2) \reduce S.O_1 + S.O_2
\end{align*}

\subsubsection*{\textsf{OPT-ADD}}
\begin{align*}
  & \textsc{(AddZero)} && \Gamma \vdash O + \zeroK{\tau, \sigma} \reduce O \\
  & \textsc{(Fac0)} && \Gamma \vdash O + O \reduce (1 + 1).O \\
  & \textsc{(Fac1)} && \Gamma \vdash S.O + O \reduce (S + 1).O \\
  & \textsc{(Fac2)} && \Gamma \vdash S_1.O + S_2.O \reduce (S_1 + S_2).O
\end{align*}

\subsubsection*{\textsf{OPT-MUL}}
\begin{align*}
  & \textsc{(MulZeroL)} &&
  \Gamma \vdash \mathbf{0}_\mathcal{O} \cdot O \reduce \mathbf{0}_\mathcal{O}
  \qquad
  \textcolor{red}{\Gamma \vdash \zeroO{\tau, \sigma} \cdot O \reduce \zeroO{\tau, \rho}\ \textrm{(where $\Gamma \vdash O : \mathcal{O}(\sigma, \rho)$)}} \\
  & \textsc{(MulZeroR)} && 
  \Gamma \vdash O \cdot \mathbf{0}_\mathcal{O} \reduce \mathbf{0}_\mathcal{O}
  \qquad
  \textcolor{red}{\Gamma \vdash O \cdot \zeroO{\sigma, \rho} \reduce \zeroO{\tau, \rho}\ \textrm{(where $\Gamma \vdash O : \mathcal{O}(\tau, \sigma)$)}} \\
  & \textsc{(REFFF)} && \Gamma \vdash (K \otimes B) \cdot O \reduce K \otimes (B \cdot O)\\
  & \textsc{(REFFF)} && \Gamma \vdash O \cdot (K \otimes B) \reduce (O \cdot K) \otimes B\\
  & \textsc{(MulScrL)} && \Gamma \vdash (S.O_1) \cdot O_2 \reduce S.(O_1 \cdot O_2) \\
  & \textsc{(MulScrR)} && \Gamma \vdash O_1 \cdot (S.O_2) \reduce S.(O_1 \cdot O_2) \\
  & \textsc{(MulDistL)} && \Gamma \vdash (O_1 + O_2) \cdot O_3 \reduce O_1 \cdot O_3 + O_2 \cdot O_3 \\
  & \textsc{(MulDistR)} && \Gamma \vdash O_1 \cdot (O_2 + O_3) \reduce O_1 \cdot O_2 + O_1 \cdot O_3 \\
  & \textsc{(MulAssoc)} && \Gamma \vdash (O_1 \cdot O_2) \cdot O_3 \reduce O_1 \cdot (O_2 \cdot O_3) \\
  & \textsc{(ReFac2)} && \Gamma \vdash (O_1 \otimes O_2) \cdot (O_1' \otimes O_2') \reduce (O_1 \cdot O_1') \otimes (O_2 \cdot O_2') \\
  & \textsc{(MulAssoc2)} && \Gamma \vdash (O_1 \otimes O_2) \cdot ((O_1' \otimes O_2') \cdot O_3) \reduce ((O_1 \cdot O_1') \otimes (O_2 \cdot O_2')) \cdot O_3 \\
\end{align*}

\textbf{Remark:} The \textsc{(MulAssoc2)} rule is for completion.


\subsubsection*{\textsf{OPT-TSR}}
\begin{align*}
  & \textsc{(TsrZero)} && \Gamma \vdash \mathbf{0}_\mathcal{O} \otimes O \reduce \mathbf{0}_\mathcal{O}
  \qquad
  \textcolor{red}{\Gamma \vdash \zeroO{\tau, \sigma} \otimes O \reduce \zeroO{\tau * \tau', \sigma * \sigma'}\ \textrm{(where $\Gamma \vdash O : \mathcal{O}(\tau', \sigma')$)}} \\
  & \textsc{(TsrZero)} && \Gamma \vdash O \otimes \mathbf{0}_\mathcal{O} \reduce \mathbf{0}_\mathcal{O}
  \qquad
  \textcolor{red}{\Gamma \vdash O \otimes \zeroO{\tau', \sigma'} \reduce \zeroO{\tau * \tau', \sigma * \sigma'}\ \textrm{(where $\Gamma \vdash O : \mathcal{O}(\tau, \sigma)$)}}\\
  & \textsc{(TsrOuter)} && \Gamma \vdash (K_1 \otimes B_1) \otimes (K_2 \otimes B_2) \reduce (K_1 \otimes K_2) \otimes (B_1 \otimes B_2) \\
  & \textsc{(TsrScrL)} && \Gamma \vdash (S.O_1) \otimes O_2 \reduce S.(O_1 \otimes O_2) \\
  & \textsc{(TsrScrR)} && \Gamma \vdash O_1 \otimes (S.O_2) \reduce S.(O_1 \otimes O_2) \\
  & \textsc{(TsrDistL)} && \Gamma \vdash (O_1 + O_2) \otimes O_3 \reduce O_1 \otimes O_3 + O_2 \otimes O_3 \\
  & \textsc{(TsrDistR)} && \Gamma \vdash O_1 \otimes (O_2 + O_3) \reduce O_1 \otimes O_2 + O_1 \otimes O_3 \\
\end{align*}






\begin{lemma}[preservation of Dirac types]
  For any context $\Gamma$ and terms $e, e'$, if $\Gamma \vdash e : \tau$ and $\Gamma \vdash e \reduce e'$, we have
  $ \Gamma \vdash e' : \tau $. 
\end{lemma}
\begin{proof} TO BE PROVED
\end{proof}


\subsection{Termination and Confluence}

\begin{proposition}
  The rules of \textsf{BASIS} are terminating and confluent.
\end{proposition}
\begin{proof}
  The three rules follow the simplification order, therefore are terminating. For confluence, the only critical pairs are
  \begin{gather*}
  (t_1, \snd\ (t_1, t_2)) \reducefrom (\fst\ (t_1, t_2), \snd\ (t_1, t_2)) \reduce (t_1, t_2) \\
  (\fst\ (t_1, t_2), t_2) \reducefrom (\fst\ (t_1, t_2), \snd\ (t_1, t_2)) \reduce (t_1, t_2) 
  \end{gather*}
  which are both joinable.
\end{proof}


\subsection{Refinement of the Reduction Rules}

\subsubsection*{\textsf{SCAL-DELTA*}}
\begin{align*}
  & \textsc{(DeltaDcp)} && 
  \Gamma \vdash \delta_{u, (s, t)} \reduce \delta_{\fst\ u, s} \times \delta_{\snd\ u, t} \\
  & \textsc{(DeltaPair)} &&
  \Gamma \vdash \delta_{\fst\ u, \fst\ v}\times\delta_{\snd\ u, \snd\ v} \reduce \delta_{u, v}\\
  \\ 
  & \textsc{(Delta1)} && 
  \Gamma \vdash \delta_{s, s} \reduce 1
  \\
  \\
  & \textsc{(Delta0)} && 
  \frac{\bigwedge_i s_i =_\textsf{BASIS} t_i \textrm{ is not satisfiable}}{\Gamma \vdash \prod_i \delta_{s_i, t_i} \reduce 0}
\end{align*}