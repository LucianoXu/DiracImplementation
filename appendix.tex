


\section{Other Examples}

This section collects typical examples for Dirac notation syntax and equality.

\begin{example}
  $$
  \sum_{k=0}^{N-1} e^{\frac{2\pi i j k}{N}}\ket{k}
  $$
  Assume $t$ and $b$ are bit strings.
  $$
  \frac{1}{\sqrt{2^n}} \sum_t (-1)^{\sum_i b_i \cdot t_i} \ket{t}_{\bar{x}}
  $$\end{example}
This example requires
\begin{itemize}
  \item big-op of sum (index in vector term and coefficent).
\end{itemize}
This example clearly shows that the linear algebra module is dependent on the quantum term and type. For example, here $\sum_i b_i \cdot t_i$ should be considered as a quantum term, but the whole expression $(-1)^{\sum_i b_i \cdot t_i}$ is a complex number term.



\begin{example}
  $$
  \bigotimes_{i = 0}^{k-1} \ket{0}_{x_i}
  $$
  $$
  \left ( \bigotimes_{i = 0}^{k-1} \texttt{QFT}[x_i] \right ) \ket{0}_{\bar{x}}
  $$
\end{example}
This example requires 
\begin{itemize}
  \item tuple of quantum variable, variable indices,
  \item big-op of tensor (index in quantum tuple).
\end{itemize}

\begin{example}
  Assume $U$ is a unitary transform,
  $$
  \frac{1}{\sqrt{2^t}} \sum_{j=0}^{2^t-1} \ket{j} U^j \ket{u}
  $$
\end{example}
This example requires
\begin{itemize}
  \item integer index and power of operators,
  \item big-op of sum (index in the exponent).
\end{itemize}

\begin{example}
  $$
  \sum_{s=0}^{r-1} \sum_{j=0}^{2^t-1} e^{\frac{2\pi i s j}{r}} \ket{j} \ket{u_s}
  $$
\end{example}
This example requires
\begin{itemize}
  \item nested big-op of sum.
\end{itemize}


\subsection{Equality Benchmark}

\begin{example}
  $$
  (U_{S_1} \otimes I_{S_2}) (\ket{\phi}_{S_1} \ket{\psi}_{S_2}) = (U_{S_1} \ket{\phi}_{S_1}) (I_{S_2}\ket{\psi}_{S_2}) = (U_{S_1} \ket{\phi}_{S_1}) \ket{\psi}_{S_2}
  $$
\end{example}
This example requires 
\begin{itemize}
  \item type variables, the universal quantification on type $T$:
    $$
    \forall (T : \texttt{qType}) (U : (T, T)) (S_1 : T), \dots
    $$
  \item quantum variables,
  \item vector and linear operator variables.
\end{itemize}

\begin{example} [ParaHadamard]
  $$
  (\bigotimes_i^n H_{x_i}) \ket{b}_{\bar{x}} = \frac{1}{\sqrt{2^n}} \sum_t (-1)^{\sum_i b_i t_i} \ket{t}_{\bar{x}}
  $$
\end{example}

