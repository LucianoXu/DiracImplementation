
\section{STLC with Product and Projection}

\begin{definition}
    A \textbf{simply typed lambda calculus} with product types and projections consists of types $\tau$ and terms $e$. The syntax is:
    \begin{align*}
        \tau ::=&\ T\ |\ \tau \to \tau\ |\ \texttt{Unit}\ |\ \tau * \tau \\
        e ::=&\ x\ |\ c\ |\ \lambda x : \tau. e\ |\ e\ e\ |\ ()\ |\ (e, e)\ |\ \pi_1\ e\ |\ \pi_2\ e
    \end{align*}
    Here $T \in B$ is a basic type, $x$ is a variable and $c$ is a constant.
\end{definition}

\begin{definition}[typing rules]
    A typing assumption has the form $x : \tau$, meaning variable $x$ has the type $\tau$. A typing context $\Gamma$ consists of typing assumptions and each variable appears only once at most.

    A typing judgement $\Gamma \vdash e : \sigma$ indicates that $e$ is a term of type $\sigma$ in context $\Gamma$. The well-typed lambda terms are defined by the following rules:
    \begin{gather*}
        \frac{x : \sigma \in \Gamma}{\Gamma \vdash x : \sigma}
        \qquad \frac{c\ \textrm{is a constant of}\ T}{\Gamma \vdash c : T}\\
        \ \\
        \frac{\Gamma::(x : \tau) \vdash e : \sigma}{\Gamma \vdash (\lambda x : \tau.e) : (\tau \to \sigma)}
        \qquad \frac{\Gamma \vdash e_1 : \tau \to \sigma \qquad \Gamma \vdash e_2 : \tau}{\Gamma \vdash e_1\ e_2 : \sigma}\\
        \ \\
        \frac{}{\Gamma \vdash () : \texttt{Unit}}
        \qquad \frac{\Gamma \vdash e_1 : \tau \qquad \Gamma \vdash e_2 : \sigma}{\Gamma \vdash (e_1, e_2) : \tau * \sigma}
        \qquad \frac{\Gamma \vdash e : \tau * \sigma}{\Gamma \vdash \pi_1\ e : \tau}
        \qquad \frac{\Gamma \vdash e : \tau * \sigma}{\Gamma \vdash \pi_2\ e : \sigma}
    \end{gather*}
\end{definition}

\begin{definition}[equational theory]
    The reduction rules for terms are:
    \begin{gather*}
        \frac{\Gamma::(x:\tau)\vdash t:\sigma\qquad \Gamma\vdash u:\tau}{\Gamma \vdash (\lambda x : \tau.t)u \ \triangleright_\beta\ t[x/u]}
        \qquad 
        \frac{\Gamma \vdash t : \tau \to \sigma\qquad x\ \textrm{is free in}\ t}{\Gamma \vdash \lambda x : \tau. t\ x\ \triangleright_\eta\ t}\\
        \ \\
        \frac{\Gamma \vdash (s, t) : \sigma * \tau}{\Gamma \vdash \pi_1\ (s, t)\ \triangleright_\pi\ s}
        \qquad
        \frac{\Gamma \vdash (s, t) : \sigma * \tau}{\Gamma \vdash \pi_2\ (s, t)\ \triangleright_\pi\ t}
        \qquad
        \frac{\Gamma \vdash u : \sigma * \tau}{\Gamma \vdash (\pi_1\ u, \pi_2\ u)\ \triangleright_\pi\ u}
        \qquad
        \frac{\Gamma \vdash t : \texttt{Unit}}{\Gamma \vdash t\ \triangleright_\pi\ ()}
    \end{gather*}

    Two terms $s$ and $t$ are $\beta\eta\pi$-equivalent in context $\Gamma$, written as $\Gamma \vdash s =_{\beta\eta\pi} t$, if they have the same normal form after $\beta\eta\pi$-reduction.
\end{definition}

Due to the product and unit types, we need to extend the $\beta\eta\pi$-equivalence to contain the isomorphism as well. This is necessary because the linear space follows the same isomorphism.
\begin{definition}[equational theory with isomorphism]
    The reduction rules incorporating isomorphism are:
    \begin{gather*}
        \texttt{Unit} * \tau\ \triangleright_\phi\ \tau
        \qquad 
        \tau * \texttt{Unit}\ \triangleright_\phi\ \tau\\
        \ \\        
        \frac{\Gamma \vdash s : \tau}{\Gamma \vdash ((), s)\ \triangleright_\phi\ s}
        \qquad
        \frac{\Gamma \vdash s : \tau}{\Gamma \vdash (s, ())\ \triangleright_\phi\ s}
    \end{gather*}
    The first two rules defines the equational theory for types, and the corresponding extra typing rule is:
    $$
    \frac{\Gamma \vdash s : \tau\qquad \tau =_\phi \sigma}{\Gamma \vdash s : \sigma}
    $$
    Two terms are equivalent if they have the same normal form after $\beta\eta\pi\phi$-reduction.
\end{definition}

\yx{The congruence of the reductions needs to be clarified. This should have been well studied already.}



\section{Other Examples}

This section collects typical examples for Dirac notation syntax and equality.

\begin{example}
  $$
  \sum_{k=0}^{N-1} e^{\frac{2\pi i j k}{N}}\ket{k}
  $$
  Assume $t$ and $b$ are bit strings.
  $$
  \frac{1}{\sqrt{2^n}} \sum_t (-1)^{\sum_i b_i \cdot t_i} \ket{t}_{\bar{x}}
  $$\end{example}
This example requires
\begin{itemize}
  \item big-op of sum (index in vector term and coefficent).
\end{itemize}
This example clearly shows that the linear algebra module is dependent on the quantum term and type. For example, here $\sum_i b_i \cdot t_i$ should be considered as a quantum term, but the whole expression $(-1)^{\sum_i b_i \cdot t_i}$ is a complex number term.



\begin{example}
  $$
  \bigotimes_{i = 0}^{k-1} \ket{0}_{x_i}
  $$
  $$
  \left ( \bigotimes_{i = 0}^{k-1} \texttt{QFT}[x_i] \right ) \ket{0}_{\bar{x}}
  $$
\end{example}
This example requires 
\begin{itemize}
  \item tuple of quantum variable, variable indices,
  \item big-op of tensor (index in quantum tuple).
\end{itemize}

\begin{example}
  Assume $U$ is a unitary transform,
  $$
  \frac{1}{\sqrt{2^t}} \sum_{j=0}^{2^t-1} \ket{j} U^j \ket{u}
  $$
\end{example}
This example requires
\begin{itemize}
  \item integer index and power of operators,
  \item big-op of sum (index in the exponent).
\end{itemize}

\begin{example}
  $$
  \sum_{s=0}^{r-1} \sum_{j=0}^{2^t-1} e^{\frac{2\pi i s j}{r}} \ket{j} \ket{u_s}
  $$
\end{example}
This example requires
\begin{itemize}
  \item nested big-op of sum.
\end{itemize}


\subsection{Equality Benchmark}

\begin{example}
  $$
  (U_{S_1} \otimes I_{S_2}) (\ket{\phi}_{S_1} \ket{\psi}_{S_2}) = (U_{S_1} \ket{\phi}_{S_1}) (I_{S_2}\ket{\psi}_{S_2}) = (U_{S_1} \ket{\phi}_{S_1}) \ket{\psi}_{S_2}
  $$
\end{example}
This example requires 
\begin{itemize}
  \item type variables, the universal quantification on type $T$:
    $$
    \forall (T : \texttt{qType}) (U : (T, T)) (S_1 : T), \dots
    $$
  \item quantum variables,
  \item vector and linear operator variables.
\end{itemize}

\begin{example} [ParaHadamard]
  $$
  (\bigotimes_i^n H_{x_i}) \ket{b}_{\bar{x}} = \frac{1}{\sqrt{2^n}} \sum_t (-1)^{\sum_i b_i t_i} \ket{t}_{\bar{x}}
  $$
\end{example}

