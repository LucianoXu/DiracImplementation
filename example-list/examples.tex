%%
%% This is file `sample-acmlarge.tex',
%% generated with the docstrip utility.
%%
%% The original source files were:
%%
%% samples.dtx  (with options: `all,journal,bibtex,acmlarge')
%% 
%% IMPORTANT NOTICE:
%% 
%% For the copyright see the source file.
%% 
%% Any modified versions of this file must be renamed
%% with new filenames distinct from sample-acmlarge.tex.
%% 
%% For distribution of the original source see the terms
%% for copying and modification in the file samples.dtx.
%% 
%% This generated file may be distributed as long as the
%% original source files, as listed above, are part of the
%% same distribution. (The sources need not necessarily be
%% in the same archive or directory.)
%%
%%
%% Commands for TeXCount
%TC:macro \cite [option:text,text]
%TC:macro \citep [option:text,text]
%TC:macro \citet [option:text,text]
%TC:envir table 0 1
%TC:envir table* 0 1
%TC:envir tabular [ignore] word
%TC:envir displaymath 0 word
%TC:envir math 0 word
%TC:envir comment 0 0
%%
%%
%% The first command in your LaTeX source must be the \documentclass
%% command.
%%
%% For submission and review of your manuscript please change the
%% command to \documentclass[manuscript, screen, review]{acmart}.
%%
%% When submitting camera ready or to TAPS, please change the command
%% to \documentclass[sigconf]{acmart} or whichever template is required
%% for your publication.
%%
%%
\documentclass[manuscript, review, timestamp]{acmart}




\usepackage{color}
\usepackage{amsmath}
%\usepackage{amssymb}
\usepackage{graphicx}
\usepackage{amsthm}
\usepackage{stmaryrd}
\usepackage[all]{xy}
\usepackage{multirow}
\usepackage{paralist}
\usepackage{hhline}
\usepackage{bm}
\usepackage{braket}
\usepackage{pifont}% http://ctan.org/pkg/pifont
\newcommand{\cmark}{\CIRCLE}%
\newcommand{\xmark}{\Circle}%
\newcommand{\hmark}{\LEFTcircle}%
\renewcommand{\matrix}[1]{\begin{bmatrix}#1\end{bmatrix}}
\usepackage{wasysym}
\usepackage{extarrows}
\usepackage{tikz}
\usetikzlibrary{positioning, shapes.geometric, graphs}
\usepackage{wrapfig}
\usepackage{dsfont}




%%
%% \BibTeX command to typeset BibTeX logo in the docs
\AtBeginDocument{%
  \providecommand\BibTeX{{%
    Bib\TeX}}}

%% Rights management information.  This information is sent to you
%% when you complete the rights form.  These commands have SAMPLE
%% values in them; it is your responsibility as an author to replace
%% the commands and values with those provided to you when you
%% complete the rights form.

%%
%% Submission ID.
%% Use this when submitting an article to a sponsored event. You'll
%% receive a unique submission ID from the organizers
%% of the event, and this ID should be used as the parameter to this command.
%%\acmSubmissionID{123-A56-BU3}

%%
%% For managing citations, it is recommended to use bibliography
%% files in BibTeX format.
%%
%% You can then either use BibTeX with the ACM-Reference-Format style,
%% or BibLaTeX with the acmnumeric or acmauthoryear sytles, that include
%% support for advanced citation of software artefact from the
%% biblatex-software package, also separately available on CTAN.
%%
%% Look at the sample-*-biblatex.tex files for templates showcasing
%% the biblatex styles.
%%

%%
%% The majority of ACM publications use numbered citations and
%% references.  The command \citestyle{authoryear} switches to the
%% "author year" style.
%%
%% If you are preparing content for an event
%% sponsored by ACM SIGGRAPH, you must use the "author year" style of
%% citations and references.
%% Uncommenting
%% the next command will enable that style.
%%\citestyle{acmauthoryear}

\newcommand{\yx}[1]{\textit{\color{blue}[YX] : #1}}

%%
%% end of the preamble, start of the body of the document source.
\begin{document}

%%
%% The "title" command has an optional parameter,
%% allowing the author to define a "short title" to be used in page headers.
\title{Example List for Dirac Notations}

%%
%% The "author" command and its associated commands are used to define
%% the authors and their affiliations.
%% Of note is the shared affiliation of the first two authors, and the
%% "authornote" and "authornotemark" commands
%% used to denote shared contribution to the research.

%%
%% By default, the full list of authors will be used in the page
%% headers. Often, this list is too long, and will overlap
%% other information printed in the page headers. This command allows
%% the author to define a more concise list
%% of authors' names for this purpose.

%%
%% The abstract is a short summary of the work to be presented in the
%% article.
\begin{abstract}
\end{abstract}

%%
%% The code below is generated by the tool at http://dl.acm.org/ccs.cfm.
%% Please copy and paste the code instead of the example below.
%%

%%
%% Keywords. The author(s) should pick words that accurately describe
%% the work being presented. Separate the keywords with commas.

%%
%% This command processes the author and affiliation and title
%% information and builds the first part of the formatted document.
\maketitle


%%%%%%%%%%%%%%%%%%%%%%%%%%%%%%%%%%%%%%%%%%%%%%%%%%%%%%%%%%%%%%%%%%%%%%%%%

\section{Textbook Examples}

\begin{example}[QCQI (2.10)]
  $$
  A\left( \sum_i a_i \ket{v_i} \right) = \sum_i a_i A(\ket{v_i}).
  $$
\end{example}

\begin{example}[QCQI (2.13)]
  $$
  \left( \ket{v}, \sum_i \lambda_i \ket{w_i} \right) = \sum_i \lambda_i (\ket{v}, \ket{w_i}).
  $$
  $$
  (\ket{v}, \ket{w}) = (\ket{w}, \ket{v})^*.
  $$
\end{example}

\begin{example}[QCQI Exercise 2.6]
  $$
  \left( \sum_i \lambda_i \ket{w_i}, \ket{v} \right) = \sum_i \lambda_i^* (\ket{w_i}, \ket{v}).
  $$
\end{example}

\begin{example}[QCQI (2.18)]
  For orthonormal basis $\ket{i}$,
  $$
  \left( \sum_i v_i \ket{i}, \sum_j w_j \ket{j} \right) = \sum_{ij}v_i^*w_j \delta_{ij} = \sum_i v_i^* w_i.
  $$
\end{example}

\begin{example}[QCQI (2.21)]
  $$
  \left( \sum_i \ket{i}\bra{i} \right) \ket{v} = \sum_i \ket{i} \braket{i|v}.
  $$
  Notice the slippery informal expression here. $\ket{v}$ should actually be $v$.
\end{example}

\begin{example}[QCQI (2.22)]
  $$
  \sum_i\ket{i}\bra{i} = I.
  $$
  (In our language, identity operator $I$ will be a symbol.a)
\end{example}


\begin{example}[QCQI (2.24-2.25)]
  $$
  \sum_{ij} \ket{w_j}\bra{i_j}A\ket{v_i}\bra{v_i} = \sum_{ij}\bra{w_j}A\ket{v_i}\ket{w_j}\bra{v_i}.
  $$
\end{example}

\begin{example}[QCQI (2.26)]
  $$
  \braket{v|v}\braket{w|w} = \sum_i\braket{v|i}\braket{i|v}\braket{w|w}.
  $$
\end{example}

\begin{example}[QCQI Exercise 2.14]
  $$
  \left( \sum_i a_i A_i \right)^\dagger = \sum_i a_i^* A_i^\dagger.
  $$
\end{example}

\begin{example}[QCQI Exercise 2.16]
  Show that any projector $P\equiv \sum_{i=1}^k \ket{i}\bra{i}$ satisfies the equation $P^2=P$.
\end{example}

\begin{example}[QCQI (2.46)]
  $$
    (A \otimes B)\left(\sum_i a_i \ket{v_i}\otimes\ket{w_i}\right) \equiv \sum_i a_i A \ket{v_i} \otimes B\ket{w_i}.
  $$
\end{example}

\begin{example}[QCQI (2.49)]
  $$
    \left(\sum_i a_i \ket{v_i}\otimes\ket{w_i}, \sum_j b_j \ket{v'_j}\otimes \ket{w'_j}\right) \equiv \sum_{ij}a_i^*b_j\braket{v_i|v'_j}\braket{w_i|w'_j}.
  $$
\end{example}

\begin{example}[QCQI (2.61)]
  $$
    \sum_i\bra{i}A\ket{\psi}\braket{\psi|i} = \bra{\psi}A\ket{\psi}.
  $$
\end{example}

\begin{example}[QCQI Exercise 2.43]
  Show that for $j,k = 1,2,3$,
  $$
    \sigma_j\sigma_k = \delta_{jk}I + i\sum_{l=1}^3\epsilon_{jkl}\sigma_l.
  $$
  This can be checked with the help of Mathematica.
\end{example}

\begin{example}[QCQI (2.128)]
  $$
    \sum_{m',m''}\bra{\psi}M_m^\dagger\bra{m'}(I_Q\otimes\ket{m}\bra{m})M_{m''}\ket{\psi}\ket{m''} = \bra{\psi}M_m^\dagger M_m\ket{\psi}.
  $$
\end{example}

\begin{example}[QCQI Theorem 4.1]
  Suppose $U$ is a unitary operation on a single qubit. Then there exist real numbers $\alpha, \beta, \gamma$ and $\delta$ such that $$
  U = e^{i\alpha}R_z(\beta)R_y(\gamma)R_z(\delta).
  $$
\end{example}

\section{PQC/Quantum Black Box Equivalence Checking}
\begin{figure*}[h]
  \includegraphics*[width=\textwidth]{QCQI.Fig4.5.png}
\end{figure*}

\begin{figure*}[h]
  \includegraphics*[width=\textwidth]{QCQI.Fig4.8.png}
\end{figure*}


\section{Operation on maximally entangled state}
\begin{example}
  Assume $S$ and $T$ are subsystems on Hilbert space $\mathcal{H}_T$ and $\ket{\Phi}_{S,T} = \sum_i \ket{i}\ket{i}$ is a maximally entangled state. Then for all operators $A\in\mathcal{L}(\mathcal{H}_T, \mathcal{H}_T)$, we have 
  $$
  A_S\ket{\Phi}_{S, T} = A_T^\top\ket{\Phi}_{S, T}.
  $$
\end{example}
\textbf{Remark:} The normal forms are 
$$
\left ( \sum_{i \in  \mathbf{U}, j \in  \mathbf{U}} ( \langle  i | \cdot   A  | j \rangle)   \left ( | i \rangle _{  S } \otimes  | j \rangle _{  T } \right ) \right )
$$
and
$$
\left ( \sum_{i \in  \mathbf{U}, j \in  \mathbf{U}}   ( \langle  i | \cdot    A^\top  | j \rangle)  \left (  | i \rangle _{  T } \otimes  | j \rangle _{  S } \right ) \right )
.
$$
The equivalence checking algorithm applies (DOT-DUAL) and (SUM-SWAP) on the terms and accepts.




\section{Semnatic Calculation of a Simple qWhile Program}
We formalized the small-step operational semantics of quantum while programs as a term rewriting system.
\begin{definition}[TRS for qWhile]
  \begin{align*}
    \langle \textbf{abort}, \rho \rangle &\longrightarrow \langle \textbf{Halt}, \textbf{0}_\mathcal{O} \rangle \\
    \langle \textbf{skip}, \rho \rangle &\longrightarrow \langle \textbf{Halt}, \rho \rangle \\
    \langle q :=0, \rho \rangle &\longrightarrow \langle \textbf{Halt}, \sum_{i \in \textbf{U}} (\ket{0}\otimes\bra{1})_{q;q} \cdot \rho \cdot (\ket{1}\otimes\bra{0})_{q;q} \rangle \\
    \langle U, \rho \rangle & \longrightarrow \langle \textbf{Halt}, U \cdot \rho \cdot U^\dagger \rangle \\
    \langle \textbf{if } P \textbf{ then } S_1 \textbf{ else } S_2 \textbf{ end}, \rho \rangle &\longrightarrow \{| \langle S_1, P \cdot \rho \cdot P \rangle, \langle S_2, (\mathbf{1}_\mathcal{O} + (-1).P) \cdot \rho \cdot (\mathbf{1}_\mathcal{O} + (-1).P)\rangle |\} \\
     & \vdots
  \end{align*}
  Note that the calculations of quantum states are represented by the syntax of Dirac notations defined above.
\end{definition}

\begin{example}
  $\langle \textbf{ while}^2\ (\ket{0}\otimes\bra{0})_q \textbf{ do } X_q\ \textbf{end}, (\ket{0}\otimes\bra{0})_q \rangle \longrightarrow \langle \textbf{ Halt}, (\ket{1}\otimes\bra{1})_q \rangle$
\end{example}

\textbf{Remark:} It should take 140 steps to complete the calculation!



\bibliographystyle{plain}
\bibliography{ref}

\end{document}
\endinput