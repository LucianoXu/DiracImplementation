%%
%% This is file `sample-acmlarge.tex',
%% generated with the docstrip utility.
%%
%% The original source files were:
%%
%% samples.dtx  (with options: `all,journal,bibtex,acmlarge')
%% 
%% IMPORTANT NOTICE:
%% 
%% For the copyright see the source file.
%% 
%% Any modified versions of this file must be renamed
%% with new filenames distinct from sample-acmlarge.tex.
%% 
%% For distribution of the original source see the terms
%% for copying and modification in the file samples.dtx.
%% 
%% This generated file may be distributed as long as the
%% original source files, as listed above, are part of the
%% same distribution. (The sources need not necessarily be
%% in the same archive or directory.)
%%
%%
%% Commands for TeXCount
%TC:macro \cite [option:text,text]
%TC:macro \citep [option:text,text]
%TC:macro \citet [option:text,text]
%TC:envir table 0 1
%TC:envir table* 0 1
%TC:envir tabular [ignore] word
%TC:envir displaymath 0 word
%TC:envir math 0 word
%TC:envir comment 0 0
%%
%%
%% The first command in your LaTeX source must be the \documentclass
%% command.
%%
%% For submission and review of your manuscript please change the
%% command to \documentclass[manuscript, screen, review]{acmart}.
%%
%% When submitting camera ready or to TAPS, please change the command
%% to \documentclass[sigconf]{acmart} or whichever template is required
%% for your publication.
%%
%%
\documentclass[manuscript, review, timestamp]{acmart}




\usepackage{color}
\usepackage{amsmath}
%\usepackage{amssymb}
\usepackage{graphicx}
\usepackage{amsthm}
\usepackage{stmaryrd}
\usepackage[all]{xy}
\usepackage{multirow}
\usepackage{paralist}
\usepackage{hhline}
\usepackage{bm}
\usepackage{braket}
\usepackage{pifont}% http://ctan.org/pkg/pifont
\newcommand{\cmark}{\CIRCLE}%
\newcommand{\xmark}{\Circle}%
\newcommand{\hmark}{\LEFTcircle}%
\renewcommand{\matrix}[1]{\begin{bmatrix}#1\end{bmatrix}}
\usepackage{wasysym}
\usepackage{extarrows}
\usepackage{tikz}
\usetikzlibrary{positioning, shapes.geometric, graphs}
\usepackage{wrapfig}
\usepackage{dsfont}

\newcommand{\tr}{\mathrm{tr}}
\newcommand{\fst}{\textrm{fst }}
\newcommand{\snd}{\textrm{snd }}
\newcommand{\pass}{\textcolor{blue}{\textbf{ [PASS] }}}
\newcommand{\fail}{\textcolor{red}{\textbf{ [FAIL] }}}


%%
%% \BibTeX command to typeset BibTeX logo in the docs
\AtBeginDocument{%
  \providecommand\BibTeX{{%
    Bib\TeX}}}

%% Rights management information.  This information is sent to you
%% when you complete the rights form.  These commands have SAMPLE
%% values in them; it is your responsibility as an author to replace
%% the commands and values with those provided to you when you
%% complete the rights form.

%%
%% Submission ID.
%% Use this when submitting an article to a sponsored event. You'll
%% receive a unique submission ID from the organizers
%% of the event, and this ID should be used as the parameter to this command.
%%\acmSubmissionID{123-A56-BU3}

%%
%% For managing citations, it is recommended to use bibliography
%% files in BibTeX format.
%%
%% You can then either use BibTeX with the ACM-Reference-Format style,
%% or BibLaTeX with the acmnumeric or acmauthoryear sytles, that include
%% support for advanced citation of software artefact from the
%% biblatex-software package, also separately available on CTAN.
%%
%% Look at the sample-*-biblatex.tex files for templates showcasing
%% the biblatex styles.
%%

%%
%% The majority of ACM publications use numbered citations and
%% references.  The command \citestyle{authoryear} switches to the
%% "author year" style.
%%
%% If you are preparing content for an event
%% sponsored by ACM SIGGRAPH, you must use the "author year" style of
%% citations and references.
%% Uncommenting
%% the next command will enable that style.
%%\citestyle{acmauthoryear}

\newcommand{\yx}[1]{\textit{\color{blue}[YX] : #1}}

%%
%% end of the preamble, start of the body of the document source.
\begin{document}

%%
%% The "title" command has an optional parameter,
%% allowing the author to define a "short title" to be used in page headers.
\title{Example List for Dirac Notations}

%%
%% The "author" command and its associated commands are used to define
%% the authors and their affiliations.
%% Of note is the shared affiliation of the first two authors, and the
%% "authornote" and "authornotemark" commands
%% used to denote shared contribution to the research.

%%
%% By default, the full list of authors will be used in the page
%% headers. Often, this list is too long, and will overlap
%% other information printed in the page headers. This command allows
%% the author to define a more concise list
%% of authors' names for this purpose.

%%
%% The abstract is a short summary of the work to be presented in the
%% article.
\begin{abstract}
\end{abstract}

%%
%% The code below is generated by the tool at http://dl.acm.org/ccs.cfm.
%% Please copy and paste the code instead of the example below.
%%

%%
%% Keywords. The author(s) should pick words that accurately describe
%% the work being presented. Separate the keywords with commas.

%%
%% This command processes the author and affiliation and title
%% information and builds the first part of the formatted document.
\maketitle


%%%%%%%%%%%%%%%%%%%%%%%%%%%%%%%%%%%%%%%%%%%%%%%%%%%%%%%%%%%%%%%%%%%%%%%%%
\section{Preliminary Example}
\begin{example}
  \pass Let $\ket{\Phi} = \sum_{i \in \mathbf{U}}\ket{i, i}$ to be the maximally entangled state on Hilbert space $\mathcal{H}\times\mathcal{H}$. Then for all operators $M_1, M_2 \in \mathrm{Hom}(\mathcal{H}, \mathcal{H})$,
  $$
  \bra{\Phi} (I \otimes M_1) (M_2 \otimes I) \ket{\Phi} = \tr(M_1^\top M_2).
  $$
\end{example}

\section{Textbook Examples}

\begin{example}[QCQI (2.10)] \pass
  $$
  A\left( \sum_i a_i \ket{v_i} \right) = \sum_i a_i A(\ket{v_i}).
  $$
\end{example}

\begin{example}[QCQI (2.13)] \pass
  $$
  \left( \ket{v}, \sum_i \lambda_i \ket{w_i} \right) = \sum_i \lambda_i (\ket{v}, \ket{w_i}).
  $$
  $$
  (\ket{v}, \ket{w}) = (\ket{w}, \ket{v})^*.
  $$
\end{example}

\begin{example}[QCQI Exercise 2.6] \pass
  $$
  \left( \sum_i \lambda_i \ket{w_i}, \ket{v} \right) = \sum_i \lambda_i^* (\ket{w_i}, \ket{v}).
  $$
\end{example}

\begin{example}[QCQI (2.18)] \pass
  For orthonormal basis $\ket{i}$,
  $$
  \left( \sum_i v_i \ket{i}, \sum_j w_j \ket{j} \right) = \sum_{ij}v_i^*w_j \delta_{ij} = \sum_i v_i^* w_i.
  $$
\end{example}

\begin{example}[QCQI (2.21)] \pass
  $$
  \left( \sum_i \ket{i}\bra{i} \right) \ket{v} = \sum_i \ket{i} \braket{i|v}.
  $$
  Notice the slippery informal expression here. $\ket{v}$ should actually be $v$.
\end{example}

\begin{example}[QCQI (2.22)] \pass
  $$
  \sum_i\ket{i}\bra{i} = I.
  $$
  (In our language, identity operator $I$ will be a symbol.a)
\end{example}


\begin{example}[QCQI (2.24-2.25)] \pass
  $$
  \sum_{ij} \ket{w_j}\bra{w_j}A\ket{v_i}\bra{v_i} = \sum_{ij}\bra{w_j}A\ket{v_i}\ket{w_j}\bra{v_i}.
  $$
\end{example}

\begin{example}[QCQI (2.26)] \pass
  $$
  \braket{v|v}\braket{w|w} = \sum_i\braket{v|i}\braket{i|v}\braket{w|w}.
  $$
\end{example}
\textbf{Remark:} need to detect $\sum_{i\in\mathbf{U}}\ket{i}\bra{i} \to \mathbf{1}_\mathcal{O}$.

\begin{example}[QCQI Exercise 2.14] \pass
  $$
  \left( \sum_i a_i A_i \right)^\dagger = \sum_i a_i^* A_i^\dagger.
  $$
\end{example}

\begin{example}[QCQI Exercise 2.16] \pass
  Show that any projector $P\equiv \sum_{i=1}^k \ket{i}\bra{i}$ satisfies the equation $P^2=P$.
\end{example}

\begin{example}[QCQI (2.46)] \pass
  $$
    (A \otimes B)\left(\sum_i a_i \ket{v_i}\otimes\ket{w_i}\right) \equiv \sum_i a_i A \ket{v_i} \otimes B\ket{w_i}.
  $$
\end{example}

\begin{example}[QCQI (2.49)] \pass
  $$
    \left(\sum_i a_i \ket{v_i}\otimes\ket{w_i}, \sum_j b_j \ket{v'_j}\otimes \ket{w'_j}\right) \equiv \sum_{ij}a_i^*b_j\braket{v_i|v'_j}\braket{w_i|w'_j}.
  $$
\end{example}

\begin{example}[QCQI (2.61)] \pass
  $$
    \sum_i\bra{i}A\ket{\psi}\braket{\psi|i} = \bra{\psi}A\ket{\psi}.
  $$
\end{example}

% \begin{example}[QCQI Exercise 2.43] \fail
%   Show that for $j,k = 1,2,3$,
%   $$
%     \sigma_j\sigma_k = \delta_{jk}I + i\sum_{l=1}^3\epsilon_{jkl}\sigma_l.
%   $$
%   This can be checked with the help of Mathematica.
% \end{example}

\begin{example}[QCQI (2.128)] \pass
  $$
    \sum_{m',m''}\bra{\psi}M_m^\dagger\bra{m'}(I\otimes\ket{m}\bra{m})M_{m''}\ket{\psi}\ket{m''} = \bra{\psi}M_m^\dagger M_m\ket{\psi}.
  $$
\end{example}

\begin{example}[QCQI Theorem 4.1] \pass
  Suppose $U$ is a unitary operation on a single qubit. Then there exist real numbers $\alpha, \beta, \gamma$ and $\delta$ such that $$
  U = \left[
  \begin{array}{cc}
  e^{i(\alpha - \beta/2 - \delta/2)} \cos \frac{\gamma}{2} & -e^{i(\alpha - \beta/2 + \delta/2)} \sin \frac{\gamma}{2} \\
  e^{i(\alpha + \beta/2 - \delta/2)} \sin \frac{\gamma}{2} & e^{i(\alpha + \beta/2 + \delta/2)} \cos \frac{\gamma}{2}
  \end{array}
  \right]
  = e^{i\alpha}R_z(\beta)R_y(\gamma)R_z(\delta),
  $$
  where
\[
  R_x(\theta) \equiv
  \begin{bmatrix}
  \cos \frac{\theta}{2} & -i \sin \frac{\theta}{2} \\
  -i \sin \frac{\theta}{2} & \cos \frac{\theta}{2}
  \end{bmatrix}
  \qquad
  R_y(\theta) \equiv
  \begin{bmatrix}
  \cos \frac{\theta}{2} & -\sin \frac{\theta}{2} \\
  \sin \frac{\theta}{2} & \cos \frac{\theta}{2}
  \end{bmatrix}
  \qquad
  R_z(\theta) \equiv
  \begin{bmatrix}
  e^{-i \theta / 2} & 0 \\
  0 & e^{i \theta / 2}
  \end{bmatrix}
  \]
  are the rotation gates.
\end{example}


\section{Operation on maximally entangled state}
\begin{example} \pass
  Assume $S$ and $T$ are subsystems on Hilbert space $\mathcal{H}_T$ and $\ket{\Phi}_{S,T} = \sum_i \ket{i}\ket{i}$ is a maximally entangled state. Then for all operators $A\in\mathcal{L}(\mathcal{H}_T, \mathcal{H}_T)$, we have 
  $$
  A_S\ket{\Phi}_{S, T} = A_T^\top\ket{\Phi}_{S, T}.
  $$
\end{example}


\section{CoqQ Examples}
\begin{definition}
  Some high-level operators are encoded as follows.  
  \begin{gather*}
    \tr(A) \equiv \sum_{i\in\mathbf{U}} \bra{i} A \ket{i} \\
    SWAP(A) \equiv \sum_{i\in\mathbf{U}}\sum_{j\in\mathbf{U}}\sum_{k\in\mathbf{U}}\sum_{l\in\mathbf{U}} \bra{i, j} A \ket{k, l} \ket{j, i}\bra{l, k} \\
    \tr_1(A) \equiv \sum_{i\in\mathbf{U}}\sum_{j\in\mathbf{U}} (\sum_{k\in\mathbf{U}} \bra{k, i} A \ket{k, j}) \ket{i}\bra{j} \\
    \tr_2(A) \equiv \sum_{i\in\mathbf{U}}\sum_{j\in\mathbf{U}} (\sum_{k\in\mathbf{U}} \bra{i, k} A \ket{j, k}) \ket{i}\bra{j} \\
    so2choi(E) = \sum_{i\in\mathbf{U}}\sum_{j\in\mathbf{U}} \ket{i}\bra{j} \otimes (\sum_{k\in\mathbf{U}} E_k \ket{i}\bra{j} E_k^\dagger)
  \end{gather*}
\end{definition}

\begin{enumerate}
  \item \pass $\bra{p}A\ket{i}\braket{j|q} = \delta_{j,q}\bra{p}A\ket{i}$
  \item \pass $\bra{p}(\ket{i}\bra{j}A)\ket{q} = \delta_{i, p}\bra{j}A\ket{q}$
  \item \pass $\bra{i}M^*\ket{j} = \bra{i}M\ket{j}^*$
  \item \pass $\bra{i}M^\dagger\ket{j} = \bra{j}M\ket{i}^*$
  \item \pass $(a A + B)^* = a^* A^* + B^*$
  \item \pass $(A+B)^\dagger = A^\dagger + B^\dagger$
  \item \pass $(c A)^\dagger = c^* A^\dagger$
  \item \pass $(c A + B)^\dagger = c^* A^\dagger + B^\dagger$
  \item \pass $(A \cdot B)^\dagger = B^\dagger \cdot A^\dagger$
  \item \pass $(M^\dagger)^\dagger = M$
  \item \pass $(aI)^\dagger = a^* I$
  \item \pass $(1 I)^\dagger = I$
  \item \pass $(\ket{i}\bra{j})^* = \ket{i}\bra{j}$
  \item \pass $(\ket{i}\bra{j})^\dagger = \ket{j}\bra{i}$
  \item \pass $M^\top = (M^*)^\dagger$
  \item \pass $M^\top = (M^\dagger)^*$
  \item \pass $M^* = (M^\dagger)^\top$
  \item \pass $\tr(M^\dagger) = \tr(M)^*$
  \item \pass $\tr(M^*) = \tr(M)^*$
  \item \pass $(\bra{i}M)^\dagger = M^\dagger \ket{i}$
  \item \pass $(M \ket{i})^\dagger = \bra{i} M^\dagger$
  \item \pass $\bra{u} \left( \sum_{i} a_i \ket{i}\bra{i} \right) \ket{u} = \sum_i a_i \braket{u|i}\braket{i|u}$
  \item \pass $\bra{i}(A\cdot B^\dagger)\ket{j} = \bra{i} A (\bra{j}B)^\dagger$
  \item \pass $A \cdot B = \sum_i (A \ket{i})(\bra{i} B)$
  \item \pass $\bra{i}(\sum_{i \in \mathbf{U}} d_i \ket{i}\bra{i} A) = d_i \bra{i} A$
  \item \pass $\tr(A) = \sum_{i \in \mathbf{U}}\bra{i} A \ket{i}$
  \item \pass $\tr(A B) = \tr(B A)$ (entry expansion is critical here)
  \item \pass $\tr(c A + B) = c\ \tr(A) + \tr(B)$
  \item \pass $\tr(A^\dagger) = \tr(A)^*$
  \item \pass $\tr(A^\top) = \tr(A)$
  \item \pass $\tr(A^*) = \tr(A)^*$
  \item \pass $\tr(\ket{u}\bra{v}) = \braket{v|u}$
  \item \pass $\tr(\ket{u}\bra{v}) = \braket{v|u}$ ($u$, $v$ are arbitrary vectors)
  \item \pass $\sum_{i\in\mathbf{U}} \ket{i}\bra{i} = I$
  \item \pass $\ket{i}\bra{j}\cdot\ket{k}\bra{l} = \delta_{j,k}\ket{i}\bra{l}$
  \item \pass $\ket{i}\bra{j}\cdot\ket{j}\bra{k} = \ket{i}\bra{k}$
  \item \pass $\tr(A \ket{i}\bra{j}) = \bra{j} (A \ket{i})$
  \item \pass $A = \sum_{j\in\mathbf{U}}\sum_{i\in\mathbf{U}} \bra{i}(A\ket{j}) \ket{i}\bra{j}$
  \item \pass $\ket{v} = \sum_{i\in\mathbf{U}}(\braket{i|v})\ket{i}$
  \item \pass $(A\ket{u})\bra{v} = A(\ket{u}\bra{v})$
  \item \pass $\ket{u}(A\ket{v})^\dagger = (\ket{u}\bra{v})A^\dagger$
  \item \pass $\bra{b} = \sum_{i\in\mathbf{U}}(\braket{b|i}\bra{i})$
  \item \pass $(A\cdot B)\ v = A (B\ v)$
  \item \pass $\sum_{i}\sum_{j}F_i G_j A G_j^\dagger F_i^\dagger = \sum_i F_i (\sum_j G_j A G_j^\dagger) F_i^\dagger$
  \item \pass $(A\cdot B)\cdot C = A \cdot (B \cdot C)$
  \item \pass $\sum_i F_i \cdot (\sum_j \sum_k G_j \cdot H_k \cdot A \cdot H_k^\dagger \cdot G_j^\dagger) \cdot F_i^\dagger = \sum_i \sum_j F_i \cdot G_j \cdot (\sum_k H_k \cdot A \cdot H_k^\dagger) \cdot G_j^\dagger \cdot F_i^\dagger$
  \item \pass $A\cdot(a B + C) = a (A\cdot B) + (A\cdot C)$
  \item \pass $(A_1 + A_2)\cdot B = A_1 \cdot B + A_2 \cdot B$
  \item \pass $A \cdot (B_1 + B_2) = A \cdot B_1 + A \cdot B_2$
  \item \pass $(-A) \cdot B = - (A \cdot B)$
  \item \pass $A \cdot (-B) = - (A \cdot B)$
  \item \pass $(k A) \cdot B = k (A \cdot B)$
  \item \pass $A \cdot (k B) = k (A \cdot B)$
  \item \pass $(k A_1 + A_2) \cdot B = k (A_1 \cdot B) + A_2 \cdot B$
  \item \pass $A \cdot (k B_1 + B_2) = k (A \cdot B_1) + A \cdot B_2$
  \item \pass $A \cdot B \cdot C \cdot D = A \cdot (B \cdot C) \cdot D$
  \item \pass $\sum_i \sum_j br_{i, j} \cdot ((f_i \cdot (a X + Y)) \cdot f_i^\dagger \cdot br_{i, j}^\dagger) = \sum_i a \sum_j br_{i, j} \cdot ((f_i \cdot X) \cdot f_i^\dagger) \cdot br_{i, j}^\dagger + \sum_i \sum_j br_{i, j} \cdot ((f_i \cdot Y) \cdot f_i^\dagger) \cdot br_{i, j}^\dagger$
  \item \pass $c > 0 \to c \sum_i (f_i \cdot X \cdot f_i^\dagger) = \sum_i ((\sqrt{c} f_i \cdot X) \cdot (\sqrt{c} f_i^\dagger))$
  \item \pass $(M \otimes N)^\dagger = M^\dagger \otimes N^\dagger$
  \item \pass $\tr (M \otimes N) = \tr(M) \tr(N)$ (Index split is required)
  \item \pass $SWAP(SWAP(A))=A$
  \item \pass $A = \sum_{i \in \mathbf{U}} \bra{\fst i} A \ket{\snd i} \ket{\fst i} \bra{\snd i}$
  \item \pass $\sum_{i\in T} f_i \cdot (\sum_{j \in R} g_j \cdot X \cdot g_j^\dagger) \cdot f_i^\dagger = \sum_{i \in T\times R} f_{\fst k} \cdot g_{\snd k} \cdot X \cdot f_{\fst k}^\dagger \cdot g_{\snd k}^\dagger$
  \item \pass $SWAP(a X + Y) = a SWAP(X) + SWAP(Y)$
  \item \pass $SWAP(A\otimes B) = B \otimes A$
  \item \pass $\tr(SWAP(A)) = \tr(A)$
  \item \pass $SWAP(A \cdot B) = SWAP(A) \cdot SWAP(B)$
  \item \pass $SWAP(A)^\top = SWAP(A^\top)$
  \item \pass $SWAP(A)^\dagger = SWAP(A^\dagger)$
  \item \pass $\tr_2(A) = \tr_1(SWAP(A))$
  \item \pass $\tr_1(A) = \tr_2(SWAP(A))$
  \item \pass $\tr_2(c A + B) = c \tr_2(A) + \tr_2(B)$
  \item \pass $\tr(A) = \tr(\tr_2(A))$
  \item \pass $\tr(A) = \tr(\tr_1(A))$
  \item \pass $\tr_1 (A \cdot (I \otimes B)) = \tr_1 (A) \cdot B$
  \item \pass $I \otimes I = I$
  \item \pass $\bra{i} A \cdot B \ket{j} = \sum_{a\in\mathbf{U}} \sum_{b\in\mathbf{U}} \bra{i}A\ket{a,b}\bra{a,b}B\ket{j}$
  \item \pass $\bra{k,p} ((\ket{i}\bra{j} \otimes I) \cdot A) \ket{q} = \delta_{i, k} \bra{j, p} A \ket{q}$
  \item \pass $\bra{p} (A \cdot (\ket{i}\bra{j} \otimes I)) \ket{k, q} = \delta_{j, k} \bra{p} A \ket{i, q}$
  \item \fail $\tr_1(so2choi(E \cdot (X^\top \otimes I))) = \sum_{k\in\mathbf{U}} E_k X E_k^\dagger$ (the formalization of $E_i$ is problematic)
  \item \fail $so2choi(a X + Y) =a\ so2choi(X) + so2choi(Y)$
  \item \pass $\tr_2(A \cdot (B \otimes I)) = \tr_2(A) \cdot B$
\end{enumerate}

\clearpage
\appendix 

\section{PQC/Quantum Black Box Equivalence Checking}
\begin{figure*}[h]
  \includegraphics*[width=\textwidth]{QCQI.Fig4.5.png}
\end{figure*}

\begin{figure*}[h]
  \includegraphics*[width=\textwidth]{QCQI.Fig4.8.png}
\end{figure*}




\section{Semnatic Calculation of a Simple qWhile Program}
We formalized the small-step operational semantics of quantum while programs as a term rewriting system.
\begin{definition}[TRS for qWhile]
  \begin{align*}
    \langle \textbf{abort}, \rho \rangle &\longrightarrow \langle \textbf{Halt}, \textbf{0}_\mathcal{O} \rangle \\
    \langle \textbf{skip}, \rho \rangle &\longrightarrow \langle \textbf{Halt}, \rho \rangle \\
    \langle q :=0, \rho \rangle &\longrightarrow \langle \textbf{Halt}, \sum_{i \in \textbf{U}} (\ket{0}\otimes\bra{1})_{q;q} \cdot \rho \cdot (\ket{1}\otimes\bra{0})_{q;q} \rangle \\
    \langle U, \rho \rangle & \longrightarrow \langle \textbf{Halt}, U \cdot \rho \cdot U^\dagger \rangle \\
    \langle \textbf{if } P \textbf{ then } S_1 \textbf{ else } S_2 \textbf{ end}, \rho \rangle &\longrightarrow \{| \langle S_1, P \cdot \rho \cdot P \rangle, \langle S_2, (\mathbf{1}_\mathcal{O} + (-1).P) \cdot \rho \cdot (\mathbf{1}_\mathcal{O} + (-1).P)\rangle |\} \\
     & \vdots
  \end{align*}
  Note that the calculations of quantum states are represented by the syntax of Dirac notations defined above.
\end{definition}

\begin{example}
  $\langle \textbf{ while}^2\ (\ket{0}\otimes\bra{0})_q \textbf{ do } X_q\ \textbf{end}, (\ket{0}\otimes\bra{0})_q \rangle \longrightarrow \langle \textbf{ Halt}, (\ket{1}\otimes\bra{1})_q \rangle$
\end{example}

\textbf{Remark:} It should take 140 steps to complete the calculation!


\bibliographystyle{plain}
\bibliography{ref}

\end{document}
\endinput